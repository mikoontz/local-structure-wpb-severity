\documentclass[]{article}
\usepackage{lmodern}
\usepackage{amssymb,amsmath}
\usepackage{ifxetex,ifluatex}
\usepackage{fixltx2e} % provides \textsubscript
\ifnum 0\ifxetex 1\fi\ifluatex 1\fi=0 % if pdftex
  \usepackage[T1]{fontenc}
  \usepackage[utf8]{inputenc}
\else % if luatex or xelatex
  \ifxetex
    \usepackage{mathspec}
  \else
    \usepackage{fontspec}
  \fi
  \defaultfontfeatures{Ligatures=TeX,Scale=MatchLowercase}
\fi
% use upquote if available, for straight quotes in verbatim environments
\IfFileExists{upquote.sty}{\usepackage{upquote}}{}
% use microtype if available
\IfFileExists{microtype.sty}{%
\usepackage{microtype}
\UseMicrotypeSet[protrusion]{basicmath} % disable protrusion for tt fonts
}{}
\usepackage[margin=1in]{geometry}
\usepackage{hyperref}
\hypersetup{unicode=true,
            pdfborder={0 0 0},
            breaklinks=true}
\urlstyle{same}  % don't use monospace font for urls
\usepackage{longtable,booktabs}
\usepackage{graphicx,grffile}
\makeatletter
\def\maxwidth{\ifdim\Gin@nat@width>\linewidth\linewidth\else\Gin@nat@width\fi}
\def\maxheight{\ifdim\Gin@nat@height>\textheight\textheight\else\Gin@nat@height\fi}
\makeatother
% Scale images if necessary, so that they will not overflow the page
% margins by default, and it is still possible to overwrite the defaults
% using explicit options in \includegraphics[width, height, ...]{}
\setkeys{Gin}{width=\maxwidth,height=\maxheight,keepaspectratio}
\IfFileExists{parskip.sty}{%
\usepackage{parskip}
}{% else
\setlength{\parindent}{0pt}
\setlength{\parskip}{6pt plus 2pt minus 1pt}
}
\setlength{\emergencystretch}{3em}  % prevent overfull lines
\providecommand{\tightlist}{%
  \setlength{\itemsep}{0pt}\setlength{\parskip}{0pt}}
\setcounter{secnumdepth}{0}
% Redefines (sub)paragraphs to behave more like sections
\ifx\paragraph\undefined\else
\let\oldparagraph\paragraph
\renewcommand{\paragraph}[1]{\oldparagraph{#1}\mbox{}}
\fi
\ifx\subparagraph\undefined\else
\let\oldsubparagraph\subparagraph
\renewcommand{\subparagraph}[1]{\oldsubparagraph{#1}\mbox{}}
\fi

%%% Use protect on footnotes to avoid problems with footnotes in titles
\let\rmarkdownfootnote\footnote%
\def\footnote{\protect\rmarkdownfootnote}

%%% Change title format to be more compact
\usepackage{titling}

% Create subtitle command for use in maketitle
\newcommand{\subtitle}[1]{
  \posttitle{
    \begin{center}\large#1\end{center}
    }
}

\setlength{\droptitle}{-2em}

  \title{}
    \pretitle{\vspace{\droptitle}}
  \posttitle{}
    \author{}
    \preauthor{}\postauthor{}
    \date{}
    \predate{}\postdate{}
  
\usepackage{booktabs}
\usepackage{longtable}
\usepackage{array}
\usepackage{multirow}
\usepackage{wrapfig}
\usepackage{float}
\usepackage{colortbl}
\usepackage{pdflscape}
\usepackage{tabu}
\usepackage{threeparttable}
\usepackage{threeparttablex}
\usepackage[normalem]{ulem}
\usepackage{makecell}
\usepackage{xcolor}

\usepackage[left]{lineno}
\linenumbers
\usepackage{setspace}
\doublespacing
\DeclareUnicodeCharacter{200E}{}

\begin{document}

\section{Relative effect of host versus non-host vegetation structure on
forest insect severity depends on climatic water
deficit}\label{relative-effect-of-host-versus-non-host-vegetation-structure-on-forest-insect-severity-depends-on-climatic-water-deficit}

Michael J. Koontz\textsuperscript{1,2,*}, Andrew M.
Latimer\textsuperscript{1,2}, Leif A. Mortenson\textsuperscript{3},
Christopher J. Fettig\textsuperscript{3}, Constance I.
Millar\textsuperscript{4}, Malcolm P. North\textsuperscript{1,2,5}

\textsuperscript{1}Graduate Group in Ecology, University of Californa,
Davis, CA, USA\\
\textsuperscript{2}Department of Plant Sciences, University of
California, Davis, CA, USA\\
\textsuperscript{3}USDA Forest Service, Pacific Southwest Research
Station, Placerville, CA, USA\\
\textsuperscript{4}USDA Forest Service, Pacific Southwest Research
Station, Albany, CA, USA\\
\textsuperscript{5}USDA Forest Service, Pacific Southwest Research
Station, Davis, CA, USA

\textsuperscript{*}Correspondence: \texttt{michael.koontz@colorado.edu}

Date report generated: March 29, 2019

\subsection{Abstract}\label{abstract}

Forest insects are a primary mortality agent of trees in Sierra Nevada
mixed-conifer forests. The recent hot drought from 2012 to 2015 led to
massive tree die-off throughout the state of California, and especially
in the Sierra Nevada.

\subsection{Introduction}\label{introduction}

Aggressive bark beetles dealt the final blow to many of the nearly 150
million trees killed in the California drought of 2012 to 2015 and its
aftermath (USDAFS 2019). A harbinger of climate change effects to come,
high temperatures exacerbating the extreme drought led to tree mortality
events of unprecedented size in the driest, densest forests across the
state (Millar and Stephenson 2015, Young et al. 2017). A century of fire
suppression policy has enabled forests to grow unchecked into dense
stands, which increases water stress on trees and makes them more
vulnerable to bark beetle attack (Fettig 2012, North et al. 2015).

Previous studies show that bark beetles thrive in denser forests (Fettig
2012), but density is often a coarse gauge of the size and spatial
distribution of trees-- the forest structure-- with which bark beetles
interact (Raffa et al. 2008).

Recent research has shown a strong link between complex forest structure
and forest resilience, but measuring this complexity generally requires
expensive equipment or labor-intensive field surveys (Larson and
Churchill 2012, Kane et al. 2014). These barriers restrict survey
frequency and extent, which limits insights into phenomena like bark
beetle outbreaks that rapidly emerge over weeks to months but have
long-lasting effects on forest conditions.

Further, the clear and vast latitudinal gradient of mortality challenges
our ability to simultaneously consider how environmental conditions may
interact with local forest structure to produce patterns of insect
activity.

along a strong south to north latitudinal gradient (Young et al. 2017,
USDAFS 2019).

Latitudinal and elevational gradients in the intensity of bark beetle
activity during the recent California drought provide unique
opportunities for a postmortem analysis of a major tree die off and how
intersecting forces of forest structure and environmental conditions
affect disturbance dynamics. Quantitative, fine-scale measures of tree
condition across these geographic gradients will enable broad-scale
assessment of forest structure as well as the intensity of western pine
beetle-induced tree mortality. Combined, these measurements can better
our understanding of how complex forest structure affects insect
disturbance, and vice versa, across the Sierra Nevada. Sound forest
management requires a better understanding of the relationships between
forest spatial structure, environmental conditions, and disturbance,
which ultimately depends on accurate measurement of forest structure at
appropriate spatial scales.

\subsubsection{How forest structure affects bark beetle
activity}\label{how-forest-structure-affects-bark-beetle-activity}

Water stress and competition (Hayes et al. 2009, Young et al. 2017) Host
availability (ease of dispersal to new hosts) Beetles prefer larger
trees under outbreak conditions

Forests in California's Sierra Nevada region are characterized by
regular bark beetle disturbances that interact with forest structure.
Bark beetles shape forest structure as they sporadically kill weakened
trees under normal conditions, or wide swaths of even healthy trees
under outbreak conditions. Forest structure also strongly influences
bark beetle activity. Low-density forests are less prone to bark beetle
attacks, but resolving the mechanism underlying this observation
requires a more nuanced view of forest structure. For instance, a
low-density forest may resist attack because its trees are in smaller
clumps with greater average tree vigor, or because its wider canopy
openings disrupt pheromone signaling between beetles (Fettig 2012).
Thus, it remains poorly understood how complex forest structure affects
and is affected by bark beetle activity.

Forest spatial structure, the size and distribution of trees in the
forest, is thought to be a key determinant of forest resilience. To
date, much of the work on Sierra Nevada forest resilience focuses on
stem density, which belies the complexity of forest structure and how it
interacts with disturbance. However, complex forest structure is
challenging to quantify, as it requires labor-intensive field surveys
(e.g., to generate stem maps) or highly specialized, expensive equipment
(e.g., LiDAR). Small, unmanned aerial systems (sUAS) enable fast and
relatively cheap remote imaging over dozens of hectares of forest, which
can be used to determine both forest structure and tree condition at the
individual tree scale. Implementing photogrammetry on the collected
images can provide a rich picture of the complex, 3-dimensional forest
structure to which bark beetles respond, and equipping the sUAS with a
multispectral sensor will allow calculation of vegetation indices (e.g.,
NDVI) commonly used to assess tree condition.

Use drones for insect attack studies! (Morris et al. 2017).

Climate change mitigation strategies emphasize reducing tree densities
(North et al. 2015, Young et al. 2017), but understanding the optimal
scale and pattern of tree distribution that can mitigate bark beetle
outbreaks will be vital for predicting how California forests may
respond to these interventions. This project investigates this
relationship with the following research questions:

\begin{enumerate}
\def\labelenumi{\arabic{enumi}.}
\item
  How does local host tree density and size affect the severity of
  western pine beetle disturbance?
\item
  How does total tree density and size affect the severity of western
  pine beetle disturbance?
\item
  How does environmentally-driven tree moisture stress affect the
  severity of western pine beetle disturbance?
\item
  Do the effects of forest structure and environmental condition on
  western pine beetle disturbance interact?
\end{enumerate}

We used ultra-high resolution remote sensing data from a small,
unhumanned aerial system (sUAS, aka drone) over a network of 32 sites in
the Sierra Nevada spanning 1000m of elevation and 350km of latitude and
covering a total of 9 square kilometers of forest to ask how fine-scale
forest structure affected the probability of tree mortality during the
cumulative mortality event of 2012 to 2018.

\subsection{Methods}\label{methods}

\subsubsection{Study system}\label{study-system}

The study sites comprise mostly ponderosa pine trees, \emph{Pinus
ponderosa}, whose primary bark beetle predator in California is the
western pine beetle (WPB), \emph{Dendroctonus brevicomis}. The WPB is an
aggressive bark beetle, meaning it must attack and kill live trees in
order to successfully reproduce (Raffa et al. 2008). Pioneer WPBs
disperse to a new host tree, determine the host's susceptibility to
attack, and use pheromone signals to attract other WPBs. The attracted
WPBs mass attack the tree by boring into its inner bark, laying eggs,
and dying, leaving their offspring to develop inside the doomed tree
before themselves dispersing (Raffa et al. 2008). Small WPB populations
prefer weakened trees but large populations can overwhelm the defense
mechanisms of even healthy trees. Successful attacks on large, healthy
trees are boons to bark beetle fecundity and trigger outbreaks in which
populations explode and massive tree mortality occurs. In California,
the WPB can have 3 generations in a single year giving it a greater
potential to spread rapidly through forests than its more infamous
congener, the mountain pine beetle, \emph{Dendroctonus ponderosa} (MPB).

We built our study on 180 vegetation monitoring plots at 36 sites
established between 2016 and 2017 (Fettig et al. 2019). These
established plots are located in beetle-attacked, mixed-conifer forests
across the Eldorado, Stanislaus, Sierra and Sequoia National Forests
across an elevation gradient (3000-4000 feet, 4000-5000 feet, and 5000+
feet above sea level) and have variable forest structure and disturbance
history. Plot locations were selected specifically in areas with
\textgreater{}40\% ponderosa pine basal area and \textgreater{}10\%
ponderosa pine mortality. The 0.04ha circular plots are clustered along
transects in groups of 5, with between 80 and 200m between each plot.
All trees within the plot were assessed as dead or alive. The stem
location of all trees was mapped relative to the center of each plot
using azimuth/distance measurements. Tree identity to species and
diameter at breast height (dbh) were recorded if dbh was greater than
6.35cm. During the spring and early summer of 2018, all field plots were
revisited to assess whether dead trees had fallen.

\subsubsection{Instrumentation}\label{instrumentation}

Imagery was captured using a DJI Zenmuse X3 RGB camera (DJI 2015a) and a
Micasense RedEdge3 5-band multispectral camera (Micasense 2015). We
mounted both of these instruments simultaneously on a DJI Matrice 100
aircraft (DJI 2015b) using the DJI 3-axis stabilized gimbal for the
Zenmuse X3 camera and a Micasense angled fixed mount for the RedEdge3
camera. The gimbal and the angled fixed mount ensured both instruments
were nadir-facing during image capture. Just prior or after image
capture at each site, we calibrated the RedEdge3 camera by taking an
image of a calibration panel on the ground in full sun with known
reflectance values for each of the 5 narrow bands.

\begin{longtable}[]{@{}rlrrlr@{}}
\caption{Reflectance sensitivity of the Micasense Rededge3 camera. The
calibration panel value represents the reflectance of the calibration
panel for the given wavelength.}\tabularnewline
\toprule
Band number & Band name & Center wavelength & Band width & Wavelength
range & Panel reflectance\tabularnewline
\midrule
\endfirsthead
\toprule
Band number & Band name & Center wavelength & Band width & Wavelength
range & Panel reflectance\tabularnewline
\midrule
\endhead
1 & blue (b) & 475 & 20 & 465-485 & 0.64\tabularnewline
2 & green (g) & 560 & 20 & 550-570 & 0.64\tabularnewline
3 & red (r) & 668 & 10 & 663-673 & 0.64\tabularnewline
4 & near infrared (nir) & 840 & 40 & 820-860 & 0.60\tabularnewline
5 & red edge (re) & 717 & 10 & 712-722 & 0.63\tabularnewline
\bottomrule
\end{longtable}

\subsubsection{Flight protocol}\label{flight-protocol}

Image capture was conducted as close to solar noon as possible to
minimize shadow effects (always within 4 hours; usually within 2 hours).
Prior to the aerial survey, two strips of bright orange drop cloth
(\textasciitilde{}100cm x 15cm) were positioned as an ``X'' over the
permanent monuments marking the center of the 5 field plots from Fettig
et al. (2019).

For each of the 36 sites (containing 5 plots each), we captured imagery
over the surrounding \textasciitilde{}40 hectares of forested area using
north-south aerial transects. For XXXXX sites, we surveyed less
surrounding area in order to maintain visual and radio communication
with the aircraft during flight (Table XXXXXX).

Table XXXX. Columns: Site, forest, elevation, rep, surveyed area, survey
date

We preprogrammed transect paths using Map Pilot for DJI on iOS
(hereafter Map Pilot) (Easy 2018). All transects tracked the terrain and
their altitude remained approximately constant at 120 meters above
ground level in order to maintain consistent ground sampling distance in
the imagery. Ground level was based on a 1-arc-second digital elevation
model (Farr et al. 2007) and we implemented terrain following using Map
Pilot. For this analysis, we dropped 4 sites whose imagery was of
insufficient quality to process.

Structure from motion (SfM) processing requires highly overlapping
images, especially in densely vegetated areas. We planned transects with
90\% forward overlap and 90\% side overlap at 100 meters below the lens.
Thus, with flights being at 120 meters above ground level, we achieved
slightly higher than 90/90 overlap for objects 20 meters tall or
shorter. Overlap values were based on focal length and field of view
parameters of the Zenmuse X3 camera. Images were captured at a constant
rate of 1 image every 2 seconds for both cameras. A forward overlap of
90\% at 100 meters translates to a flight speed of approximately 6.3 m/s
and a side overlap of 90\% at 100 meters translates to transects
approximately 18 meters apart. Approximately 1900 photos were captured
over each 40 hectare survey area for each camera.

\subsubsection{Structure from motion/Photogrammetric
processing}\label{structure-from-motionphotogrammetric-processing}

We used structure from motion (SfM), aka photogrammetry, to generate
orthorectified reflectance maps, digital surface models, and dense point
clouds for each field site. We used Pix4Dmapper Cloud to process imagery
using parameters ideal for images of a densely vegetated area taken by a
multispectral camera. For three sites, we processed the RGB and the
multispectral imagery in the same project to enhance the resolution of
the dense point cloud. All SfM projects resulted in a single processing
``block,'' indicating that all images in the project were optimized and
processed together.

\subsubsection{Creating canopy height
models}\label{creating-canopy-height-models}

We classified each survey area's dense point cloud into ``ground'' and
``non-ground'' points using a cloth simulation filter algorithm (Zhang
et al. 2016) implemented in the \texttt{lidR} (Roussel et al. 2019)
package. We rasterized the ground points using the \texttt{raster}
package (Hijmans et al. 2019) to create a digital terrain model
representing the ground underneath the vegetation at 1 meter resolution.
We created a canopy height model by subtracting the digital terrain
model from the digital surface model created in Pix4Dmapper.

\subsubsection{Tree detection}\label{tree-detection}

We tested a total of 7 automatic tree detection algorithms and a total
of 177 parameter sets on the canopy height model or the dense point
cloud to locate trees within each site (Table XXXXX; algorithm, number
of parameter sets, reference). We used 3 parameter sets of a variable
window filter implmented in \texttt{ForestTools} (Plowright 2018)
including the default variable window filter function in
\texttt{ForestTools} as well as the ``pines'' and ``combined'' functions
from Popescu and Wynne (2004). We used 6 parameter sets of a local
maximum filter implemented in \texttt{lidR}. We used 131 parameter sets
of the algorithm from Li et al. (2012), which operates on the original
point cloud. These parameter sets included those from Shin et al. (2018)
and Jakubowski et al. (2013). We used 3 parameter sets of the
\texttt{watershed} algorithm implemented in \texttt{lidR}, which is a
wrapper for a function in the \texttt{EBImage} package (Pau et al.
2010). We used 3 parameter sets of \texttt{ptrees} (Vega et al. 2014)
implemented in \texttt{lidR} (Roussel et al. 2019) and
\texttt{lidRplugins} (Roussel 2019) and which operates on the raw point
cloud, without first normalizing it to height above ground level (i.e..
subtracting the ground elevation from the dense point cloud). We used
the default parameter set of the \texttt{multichm} (Eysn et al. 2015)
algorithm implmented in \texttt{lidR} (Roussel et al. 2019) and
\texttt{lidRplugins} (Roussel 2019). We used 30 parameter sets of the
experimental algorithm \texttt{lmfx} (Roussel 2019).

\subsubsection{Map ground data}\label{map-ground-data}

Each orthorectified reflectance map was inspected to locate the 5 orange
``X''s marking the center of the field plots. We were able to locate 110
out of 180 field plots and were then able to use these plots for
validation of automated tree detection algorithms. We used the
\texttt{sf} package (Pebesma et al. 2019) to convert
distance-from-center and azimuth measurements of each tree in the ground
plots to an x-y position on the SfM-derived reflectance map using the
x-y position of the orange X visible in the reflectance map as the
center.

\subsubsection{Correspondence of automatic tree detection with ground
data}\label{correspondence-of-automatic-tree-detection-with-ground-data}

We calculated 7 forest structure metrics for each field plot using the
ground data collected by Fettig et al. (2019): total number of trees,
number of trees greater than 15 meters, number of trees less than 15
meters, mean height of trees, 25\textsuperscript{th} percentile tree
height, 75\textsuperscript{th} percentile tree height, mean distance to
nearest tree neighbor, mean distance to 2\textsuperscript{nd} nearest
neighbor.

For each tree detection algorithm and parameter set described above, we
calculated the same set of 7 structure metrics within the footprint of
the validation field plots. We calculated the Pearson's correlation and
root mean square error (RMSE) between the ground data and the aerial
data for each of the 7 structure metrics for each of the 177 automatic
tree detection algorithms/parameter sets.

For each algorithm and parameter set, we calculated its performance
relative to other algorithms as whether its Pearson's correlation was
within 5\% of the highest Pearson's correlation as well as whether its
RMSE was within 5\% of the lowest RMSE. For each algorithm/parameter
set, we summed the number of forest structure metrics for which it
reached these 5\% thresholds. For automatically detecting trees across
the whole study, we selected the algorithm/parameter set that performed
well across the most number of forest metrics.

\subsubsection{Segmentation of crowns}\label{segmentation-of-crowns}

We delineated individual tree crowns with a marker controlled watershed
segmentation algorithm (Meyer and Beucher 1990) using the detected
treetops as markers implemented in the \texttt{ForestTools} package
(Plowright 2018). If the automatic segmentation algorithm failed to
generate a crown segment for a detected tree (e.g., often snags with a
very small crown footprint), a circular crown was generated with a
radius of 0.5 meters. If the segmentation generated multiple polygons
for a single detected tree, only the polygon containing the detected
tree was retained. Image overlap decreases near the edges of the overall
flight path, which reduces the quality of the SfM processing in those
areas. Thus, we excluded segmented crowns within 35 meters of the edge
of the survey area.

We used the \texttt{velox} package (Hunziker 2017) to extract all the
pixel values from the orthorectified reflectance map for each of the 5
narrow bands within each segmented crown polygon. Per pixel, we
additionally calculated the normalized difference vegetation index
(NDVI; Rouse et al. (1973)), the normalized difference red edge (NDRE;
Gitelson and Merzlyak (1994)), the red-green index (RGI; Coops et al.
(2006)), the red edge chlorophyll index (CI{[}red edge{]}; Clevers and
Gitelson (2013)), and the green chlorophyll index (CI{[}green{]};
Clevers and Gitelson (2013)). For each crown polygon, we calculated the
mean value for each raw and derived reflectance band (5 raw; 5 derived).

\subsubsection{Classification of trees}\label{classification-of-trees}

We overlaid the segmented crowns on the reflectance maps from 20 sites
spanning the latitudinal and elevational gradient in the study. Using
QGIS, we hand classified XXXX trees as live/dead and as one of 5
dominant species in the study area (\emph{Pinus ponderosa}, \emph{Pinus
lambertiana}, \emph{Abies concolor}, \emph{Calocedrus decurrens}, or
\emph{Quercus kelloggi}) using the mapped ground data as a guide.

We used all 10 mean values of the reflectance bands for each tree crown
polygon to predict whether the hand classified trees were alive or dead
using a boosted logistic regression model implemented in the
\texttt{caret} package (Kuhn 2008). For just the living trees, we
similarly used all 10 reflectance values to predict the tree species
using regularized discriminant analysis implemented in the
\texttt{caret} package, which proved to have the highest accuracy for a
training dataset (accuracy = XXXXX, kappa = XXXXX).

Finally, we used these models to classify all tree crowns in the data
set as alive or dead as well as the species of living trees.

\subsubsection{Allometric scaling of height to basal
area}\label{allometric-scaling-of-height-to-basal-area}

We converted the height of each tree determined using the canopy height
model to its basal area. Using the tree height and diameter at breast
height (DBH; breast height = 1.37m) ground data from Fettig et al.
(2019), we fit a simple linear regression to predict DBH from height for
each of the 5 dominant species. Using the model-classified tree species
of each segmented tree, we used the corresponding linear relationship
for that species to estimate the DBH given the tree's height. We then
calculated each tree's basal area, assuming no tapering from breast
height.

\subsubsection{Note on assumptions about dead
trees}\label{note-on-assumptions-about-dead-trees}

For the purposes of this study, we assumed that all dead trees were
ponderosa pine and were thus host trees. This is a reasonably good
assumption, given that Fettig et al. (2019) found that 73.4\% of the
dead trees in the coincident ground plots were ponderosa pine.

\subsubsection{Rasterizing individual tree
data}\label{rasterizing-individual-tree-data}

Because the tree detection algorithms were validated against ground data
at the plot level, we rasterized the classified trees at a spatial
resolution similar to that of the ground plots (rasterized to 20m x 20m
equalling 400 m\textsuperscript{2}; circular ground plots with 11.35m
radius equalling 404 m\textsuperscript{2}). In each raster cell, we
tallied: number of alive trees, number of dead trees, number of
ponderosa pine trees, number of non-ponderosa pine trees, basal area of
ponderosa pine trees, basal area of non-ponderosa pine trees.

\subsubsection{Environmental data}\label{environmental-data}

We used climatic water deficit (CWD) (Stephenson 1998) from the
1980-2010 mean value of the basin characterization model (Flint et al.
2013) as an integrated measure of temperature and moisture conditions
for each cell. Higher values of CWD correspond to hotter, drier
conditions and lower values correspond to cooler, wetter conditions CWD
has been shown to correlate well with broad patterns of tree mortality
in the Sierra Nevada (Young et al. 2017). We resampled the climatic
water deficit product using bilinear interpolation implemented in the
\texttt{raster} package to match the 20m x 20m spatial scale of the
other variables. We converted the CWD value for each cell into a z-score
representing that cell's deviation from the mean CWD across the climatic
range of Sierra Nevada ponderosa pine as determined from XXXXX herbarium
records described in Baldwin et al. (2017). Thus, a CWD z-score of one
would indicate that the CWD at that cell is one standard deviation
hotter/drier than the mean CWD across all geolocated herbarium records
for ponderosa pine in the Sierra Nevada.

\subsubsection{Statistical model}\label{statistical-model}

We used a generalized linear model with a zero-inflated binomial
response and a logit link to predict the probability of ponderosa pine
mortality within each raster cell as a function of the crossed effects
of ponderosa pine quadratic mean diameter and density added to the
crossed effect of overall quadratic mean diameter and density as well as
the interaction of each summand with climatic water deficit at each
site.

To measure and account for spatial autocorrelation of the bark beetle
behavioral processes underlying ponderosa mortality, we first subsampled
the data at each site to a random selection of 200, 20m x 20m cells
representing approximately 27.5\% of the surveyed area. With these
subsampled data, we included a separate exact Gaussian process term per
site of the interaction between the x- and y-position of each cell using
the \texttt{gp()} function in the \texttt{brms} package (Bürkner 2017).
The Gaussian process accounts for spatial autocorrelation in the model
by jointly estimating the spatial covariance of the response variable
with the effects of the other covariates.

\[
\begin{aligned}
y_{i,j} \sim &\ \begin{cases}
    0, & p \\
    Binom(n_i, \pi_i), & 1-p
  \end{cases} \\
logit(\pi_i) = &\ \beta_0\ + \\
& \beta_1X_{cwd, j}\ + \\
& \beta_1X_{cwd, j}(\beta_2X_{pipoQMD, i} + \beta_3X_{pipoDensity, i} + \beta_4X_{pipoQMD, i}X_{pipoDensity, i})\ + \\ 
& \beta_1X_{cwd, j}(\beta_5X_{overallQMD, i} + \beta_6X_{overallDensity, i} + \beta_7X_{overallQMD, i}X_{overallDensity, i})\ + \\
& \mathcal{GP}_j(x_i, y_i) \\
\end{aligned}
\]

Where \(y_i\) is the number of dead trees in cell \(i\), \(n_i\) is the
sum of the dead trees and live ponderosa pine trees in cell \(i\),
\(\pi_i\) is the probability of ponderosa pine tree mortality in cell
\(i\), \(p\) is the probability of there being zero dead trees in a cell
arising as a result of an unmodeled process, \(X_{cwd, j}\) is the
z-score of climatic water deficit for site \(j\), \(X_{pipoQMD, i}\) is
the scaled quadratic mean diameter of ponderosa pine in cell \(i\),
\(X_{pipoDensity, i}\) is the scaled density of ponderosa pine trees in
cell \(i\), \(X_{overallQMD, i}\) is the scaled quadratic mean diameter
of all trees in cell \(i\), \(X_{overallDensity, i}\) is the scaled
density of all trees in cell \(i\), \(x_i\) and \(y_i\) are the x- and
y- coordinates of the centroid of the cell in an EPSG3310 coordinate
reference system, and \(\mathcal{GP}_j\) represents the exact Gaussian
process describing the spatial covariance between cells at site \(j\).

We used 4 chains with 2000 iterations each (1000 warmup, 1000 samples),
and confirmed chain convergence by ensuring all \texttt{Rhat} values
were less than 1.1 (Brooks and Gelman 1998). We used posterior
predictive checks to visually confirm model performance by overlaying
the density curves of the predicted number of dead trees per cell over
the observed number (Gabry et al. 2019). We used 50 random samples from
the model fit to generate 50 density curves and ensured curves were
centered on the observed distribution, paying special attention to model
performance at capturing counts of zero.

\subsubsection{Software and data
availability}\label{software-and-data-availability}

All data are available via the Open Science Framework. Statistical
analyses were performed using the \texttt{brms} packages. With the
exception of the SfM software (Pix4Dmapper Cloud) and the GIS software
QGIS, all data carpentry and analyses were performed using \texttt{R} (R
Core Team 2018).

\subsection{Results}\label{results}

\begin{figure}
\centering
\includegraphics{../../figures/eldo_3k_1_dsm.png}
\caption{Example digital surface model (DSM) that is a direct output
from the dense point cloud generated using structure from motion (SfM)
processing. The DSM represents the ground elevation plus the vegetation
height.}
\end{figure}

\begin{figure}
\centering
\includegraphics{../../figures/eldo_3k_1_dtm.png}
\caption{Example digital terrain model (DTM) resulting from processing
the digital surface model using the cloth simulation filter algorithm
(Zhang et al. 2016), which classifies points in the dense point cloud as
``ground'' or ``not-ground'' and then interpolates the ``ground''
elevation for the rest of the dense point cloud footprint. The DTM
represents the ground elevation without any vegetation.}
\end{figure}

\begin{figure}
\centering
\includegraphics{../../figures/eldo_3k_1_chm.png}
\caption{Example canopy height model (CHM) generated by subtracting the
digital terrain model from the digital surface model. The CHM represents
the height of all of the elevation above ground level.}
\end{figure}

\subsubsection{Tree detection}\label{tree-detection-1}

We found that the experimental \texttt{lmfx} algorithm with parameter
values of \texttt{dist2d\ =\ 1} and \texttt{ws\ =\ 2.5} (Roussel et al.
2019) performed the best across 7 measures of forest structure as
measured by Pearson's correlation with ground data (Table XXXX; rows are
different forest metrics, columns are correlation and RMSE).

Asterisk indicates within 5\% of the value of the best-performing model.

\begin{longtable}[]{@{}lllrr@{}}
\caption{Correlation and differences between the best performing tree
detection algorithm (lmfx with dist2d = 1 and ws = 2.5) and the ground
data. An asterisk next to the correlation or RMSE indicates that this
value was within 5\% of the value of the best-performing
algorithm/parameter set.}\tabularnewline
\toprule
Forest structure metric & Correlation with ground & RMSE & Mean error &
Median error\tabularnewline
\midrule
\endfirsthead
\toprule
Forest structure metric & Correlation with ground & RMSE & Mean error &
Median error\tabularnewline
\midrule
\endhead
height (m); 25th percentile & 0.16 & 8.46 & -2.30 & -1.16\tabularnewline
height (m); mean & 0.29 & 7.81* & -3.43 & -2.29\tabularnewline
height (m); 75th percentile & 0.35 & 10.33* & -4.85 &
-3.98\tabularnewline
dist to 1st nearest neighbor (m) & 0.55* & 1.16* & 0.13 &
0.26\tabularnewline
dist to 2nd nearest neighbor (m) & 0.61* & 1.70* & 0.08 &
0.12\tabularnewline
dist to 3rd nearest neighbor (m) & 0.50 & 2.29 & 0.17 &
0.19\tabularnewline
total tree count & 0.67* & 8.68* & 0.37 & 2.00\tabularnewline
count of trees \textgreater{} 15m & 0.43 & 7.38 & 1.18 &
0.00\tabularnewline
count of trees \textless{} 15m & 0.58 & 8.42 & -0.66 &
2.00\tabularnewline
\bottomrule
\end{longtable}

\subsubsection{Effect of local structure on western pine beetle
severity}\label{effect-of-local-structure-on-western-pine-beetle-severity}

We found a strong main effect of climatic water deficit on the
probability of ponderosa pine mortality within each 20m x 20m cell.
Greater climatic water deficit, indicating hotter/drier conditions,
increased the probability of ponderosa pine mortality.

We also found a strong effect of ponderosa pine local density,
accounting for quadratic mean diamter with greater ponderosa pine
density increasing the probability of ponderosa pine mortality.
Conversely, we found a generally negative effect of quadratic mean
diameter of ponderosa pine on the probability of ponderosa mortality,
suggesting that the western pine beetle attacked smaller trees, on
average. There was a strong positive interaction between the climatic
water deficit and ponderosa pine quadratic mean diameter, such that
larger trees were more likely to increase the probability of ponderosa
mortality in hotter, drier sites.

We found negative main effects of overall tree density and overall
quadratic mean diameter. There was a positive interaction between these
variables, such that denser stands with larger trees did lead to greater
ponderosa pine mortality.

\subsubsection{Spatial effects}\label{spatial-effects}

We were able to calculate the length scale of the spatial
autocorrelation in the probability of ponderosa pine mortality at each
site, accounting for forest structure and environmental factors. By
fitting a separate approximate Gaussian process for each site on the
interacting variables of the x- and y- position, we measured the spatial
covariance inherent in the data, accounting for other factors.

(Seidl et al. 2015) (Preisler et al. 2017)

\subsection{Discussion}\label{discussion}

\subsubsection{Similarities and differences with Fettig et al.
(2019)}\label{similarities-and-differences-with-fettig2019}

Fettig et al. (2019) found positive relationship between number of trees
killed and: total number of trees, total basal area, stand density
index.

Fettig et al. (2019) found negative relationship between the proportion
of trees killed and: total number of trees, stand density index.

Hayes et al. (2009) and Fettig et al. (2019) found measures of host
availability explained less variation in mortality than measures of
stand density.

Negrón et al. (2009) reported positive association of probability of
ponderosa pine mortality and tree density during a drought in Arizona.

Effect of competition may be masked because drought was so extreme
Fettig et al. (2019); Floyd et al. (2009), which is perhaps why we saw a
counter-intuitive signal of increasing total basal area leading to lower
probability of ponderosa pine mortality.

\subsubsection{Broader context around field
plots}\label{broader-context-around-field-plots}

We surveyed 9 square kilometers of forest representing XXXXXX trees
along a broad gradient. Site selection and small plot size can influence
inference. For instance, Fettig et al. (2019) reported statistically
undetectable differences in overall mortality in their plot network
across 4 national forests. By expanding the hectarage surveyed by a
factor of 200, we detected dramatic differences in overall mortality.

This is about more than sample size. This is also about capturing the
local disturbance phenomenon.

\subsubsection{Closer spacing between potential host trees facilitates
dispersal}\label{closer-spacing-between-potential-host-trees-facilitates-dispersal}

If this drives mortality patterns, then we'd expect the count of
ponderosa pine trees, accounting for other variables, to have a strong
positive effect.

\subsubsection{Host preference for large
trees}\label{host-preference-for-large-trees}

If this drives mortality patterns, then we'd expect the quadratic mean
diameter of ponderosa pine trees, accounting for other variables, to
have a strong positive effect.

\subsubsection{Denser forests augment pheromone
communication}\label{denser-forests-augment-pheromone-communication}

If this drives mortality patterns, then we'd expect the count of all
trees, accounting for other variables, to have a strong positive effect.

\subsubsection{Tree crowding leads to greater average water stress per
tree}\label{tree-crowding-leads-to-greater-average-water-stress-per-tree}

If this drives mortality patterns, then we'd expect the quadratic mean
dimater of all trees, accounting for other factors, to have a strong
positive effect.

\subsubsection{Interaction between host density and host
size}\label{interaction-between-host-density-and-host-size}

A positive coefficient would indicate a combined effect of WPB
preference for large trees and nearby host availability.

\subsubsection{Interaction between all tree density and all tree
size}\label{interaction-between-all-tree-density-and-all-tree-size}

A positive coefficient would indicate a combined effect of tree crowding
and pheromone communication enhancement.

\subsubsection{Interactions with climatic water
deficit}\label{interactions-with-climatic-water-deficit}

Are any of the above mechanisms exacerbated by water stress of the
trees?

\subsubsection{Spatial effect}\label{spatial-effect}

The western pine beetle is known to exhibit strong aggregation and
anti-aggregation behavior arising from its pheromone communication, and
thus it is likely that the measured spatial covariance in this study is
attributable to the magnitude of this effect at each site.

Some studies have suggested that ``outbreak'' conditions are
distinguishable by clustered tree mortality, but this is perhaps
challenging to tease apart (Raffa et al. 2008). Our modeling framework
allows for a joint estimation of the effects of forest structure,
environmental condition, and the spatial effect. This framework would be
enhanced with confidence in individual tree level data, and a lot of it,
along with a strong gradient of environmental conditions and forest
structure.

We won't interpret this measure of contagion, because the uncertainties
in this particular study are too great (tree detection, species
classification, dead trees all assumed to be WPB hosts, didn't account
for topographic effects which could also manifest as part of this
spatial covariance process).

\subsubsection{Important considerations}\label{important-considerations}

Cumulative effect of elevated insect activity, as mortality was spread
out over 5 years and we surveyed at the end. All the detected dead trees
were considered ponderosa pine-- we know this is wrong.'

\subsubsection{Room for improvement}\label{room-for-improvement}

\begin{itemize}
\tightlist
\item
  Better geometry by using higher overlap, more spatially resolved
  images.
\item
  Better image classification and scalability by using instrumentation
  having spectral overlap with more widely deployed instrumentation
  (e.g., Landsat).
\item
  Better tree detection using
\end{itemize}

\subsubsection{Future directions}\label{future-directions}

My goal is to tease apart the relative role of environmental drivers
versus behavioral drivers of bark beetle-induced tree mortality. I think
teasing these apart will help with inference about the mechanism
underlying the effect of forest structure on disturbance severity.
Crowded forests means trees are both water stressed and are closer
targets for new attacks {[}i.e., shorter dispersal needed to attack the
next tree{]}, and I think comparing the ``voronoi polygon area'' effect
with the ``spatial covariance of mortality kernel'' effect across sites
will tell us whether it's the water stress or the smaller dispersal
requirements driving mortality patterns. A big voronoi polygon area
effect and a short covariance kernel tells us that it's a water stress
effect-- a crowded tree gets attacked regardless of whether nearby trees
were attacked. A small voronoi polygon area effect and a long covariance
kernel tells us that the mortality is patterned more based on there
being spillover from nearby attacked neighbors instead of how crowded
any given tree is. I expect we might see different relative magnitudes
of voronoi polygon area and covariance kerenel effects depending on CWD.

\subsection*{References}\label{references}
\addcontentsline{toc}{subsection}{References}

\hypertarget{refs}{}
\hypertarget{ref-baldwin2017a}{}
Baldwin, B. G., A. H. Thornhill, W. A. Freyman, D. D. Ackerly, M. M.
Kling, N. Morueta-Holme, and B. D. Mishler. 2017. Species richness and
endemism in the native flora of California. American Journal of Botany
104:487--501.

\hypertarget{ref-brooks1998}{}
Brooks, S. P., and A. Gelman. 1998. General Methods for Monitoring
Convergence of Iterative Simulations. Journal of Computational and
Graphical Statistics 7:434.

\hypertarget{ref-burkner2017}{}
Bürkner, P.-C. 2017. \textbf{Brms} : An \emph{R} Package for Bayesian
Multilevel Models Using \emph{Stan}. Journal of Statistical Software 80.

\hypertarget{ref-clevers2013}{}
Clevers, J., and A. Gitelson. 2013. Remote estimation of crop and grass
chlorophyll and nitrogen content using red-edge bands on Sentinel-2 and
-3. International Journal of Applied Earth Observation and
Geoinformation 23:344--351.

\hypertarget{ref-coops2006}{}
Coops, N. C., M. Johnson, M. A. Wulder, and J. C. White. 2006.
Assessment of QuickBird high spatial resolution imagery to detect red
attack damage due to mountain pine beetle infestation. Remote Sensing of
Environment 103:67--80.

\hypertarget{ref-dji2015}{}
DJI. 2015a. Zenmuse X3 - Creativity Unleashed.
\url{https://www.dji.com/zenmuse-x3/info}.

\hypertarget{ref-dji2015a}{}
DJI. 2015b. DJI - The World Leader in Camera Drones/Quadcopters for
Aerial Photography. \url{https://www.dji.com/matrice100/info}.

\hypertarget{ref-dronesmadeeasy2018}{}
Easy, D. M. 2018. ‎Map Pilot for DJI.
\url{https://itunes.apple.com/us/app/map-pilot-for-dji/id1014765000?mt=8}.

\hypertarget{ref-eysn2015}{}
Eysn, L., M. Hollaus, E. Lindberg, F. Berger, J.-M. Monnet, M. Dalponte,
M. Kobal, M. Pellegrini, E. Lingua, D. Mongus, and N. Pfeifer. 2015. A
Benchmark of Lidar-Based Single Tree Detection Methods Using
Heterogeneous Forest Data from the Alpine Space. Forests 6:1721--1747.

\hypertarget{ref-farr2007}{}
Farr, T. G., P. A. Rosen, E. Caro, R. Crippen, R. Duren, S. Hensley, M.
Kobrick, M. Paller, E. Rodriguez, L. Roth, D. Seal, S. Shaffer, J.
Shimada, J. Umland, M. Werner, M. Oskin, D. Burbank, and D. Alsdorf.
2007. The Shuttle Radar Topography Mission. Reviews of Geophysics 45.

\hypertarget{ref-fettig2012b}{}
Fettig, C. J. 2012. Chapter 2: Forest health and bark beetles. \emph{in}
Managing Sierra Nevada Forests. PSW-GTR-237. USDA Forest Service.

\hypertarget{ref-fettig2019}{}
Fettig, C. J., L. A. Mortenson, B. M. Bulaon, and P. B. Foulk. 2019.
Tree mortality following drought in the central and southern Sierra
Nevada, California, U.S. Forest Ecology and Management 432:164--178.

\hypertarget{ref-flint2013}{}
Flint, L. E., A. L. Flint, J. H. Thorne, and R. Boynton. 2013.
Fine-scale hydrologic modeling for regional landscape applications: The
California Basin Characterization Model development and performance.
Ecological Processes 2:25.

\hypertarget{ref-floyd2009}{}
Floyd, M. L., M. Clifford, N. S. Cobb, D. Hanna, R. Delph, P. Ford, and
D. Turner. 2009. Relationship of stand characteristics to
drought-induced mortality in three Southwestern piñonJuniper woodlands.
Ecological Applications 19:1223--1230.

\hypertarget{ref-gabry2019}{}
Gabry, J., D. Simpson, A. Vehtari, M. Betancourt, and A. Gelman. 2019.
Visualization in Bayesian workflow. Journal of the Royal Statistical
Society: Series A (Statistics in Society) 182:389--402.

\hypertarget{ref-gitelson1994}{}
Gitelson, A., and M. N. Merzlyak. 1994. Spectral Reflectance Changes
Associated with Autumn Senescence of Aesculus hippocastanum L. and Acer
platanoides L. Leaves. Spectral Features and Relation to Chlorophyll
Estimation. Journal of Plant Physiology 143:286--292.

\hypertarget{ref-hayes2009}{}
Hayes, C. J., C. J. Fettig, and L. D. Merrill. 2009. Evaluation of
Multiple Funnel Traps and Stand Characteristics for Estimating Western
Pine Beetle-Caused Tree Mortality. Journal of Economic Entomology
102:2170--2182.

\hypertarget{ref-hijmans2019}{}
Hijmans, R. J., J. van Etten, M. Sumner, J. Cheng, A. Bevan, R. Bivand,
L. Busetto, M. Canty, D. Forrest, A. Ghosh, D. Golicher, J. Gray, J. A.
Greenberg, P. Hiemstra, I. for M. A. Geosciences, C. Karney, M.
Mattiuzzi, S. Mosher, J. Nowosad, E. Pebesma, O. P. Lamigueiro, E. B.
Racine, B. Rowlingson, A. Shortridge, B. Venables, and R. Wueest. 2019.
Raster: Geographic Data Analysis and Modeling.

\hypertarget{ref-hunziker2017}{}
Hunziker, P. 2017. Velox: Fast Raster Manipulation and Extraction.

\hypertarget{ref-jakubowski2013}{}
Jakubowski, M. K., W. Li, Q. Guo, and M. Kelly. 2013. Delineating
Individual Trees from Lidar Data: A Comparison of Vector- and
Raster-based Segmentation Approaches. Remote Sensing 5:4163--4186.

\hypertarget{ref-kane2014}{}
Kane, V. R., M. P. North, J. A. Lutz, D. J. Churchill, S. L. Roberts, D.
F. Smith, R. J. McGaughey, J. T. Kane, and M. L. Brooks. 2014. Assessing
fire effects on forest spatial structure using a fusion of Landsat and
airborne LiDAR data in Yosemite National Park. Remote Sensing of
Environment 151:89--101.

\hypertarget{ref-kuhn2008}{}
Kuhn, M. 2008. Building Predictive Models in R Using the caret Package.
Journal of Statistical Software 28:1--26.

\hypertarget{ref-larson2012}{}
Larson, A. J., and D. Churchill. 2012. Tree spatial patterns in
fire-frequent forests of western North America, including mechanisms of
pattern formation and implications for designing fuel reduction and
restoration treatments. Forest Ecology and Management 267:74--92.

\hypertarget{ref-li2012}{}
Li, W., Q. Guo, M. K. Jakubowski, and M. Kelly. 2012. A New Method for
Segmenting Individual Trees from the Lidar Point Cloud. Photogrammetric
Engineering \& Remote Sensing 78:75--84.

\hypertarget{ref-meyer1990}{}
Meyer, F., and S. Beucher. 1990. Morphological segmentation. Journal of
Visual Communication and Image Representation 1:21--46.

\hypertarget{ref-micasense2015}{}
Micasense. 2015. MicaSense.
\url{https://support.micasense.com/hc/en-us/articles/215261448-RedEdge-User-Manual-PDF-Download-}.

\hypertarget{ref-millar2015}{}
Millar, C. I., and N. L. Stephenson. 2015. Temperate forest health in an
era of emerging megadisturbance. Science 349:823--826.

\hypertarget{ref-morris2017}{}
Morris, J. L., S. Cottrell, C. J. Fettig, W. D. Hansen, R. L. Sherriff,
V. A. Carter, J. L. Clear, J. Clement, R. J. DeRose, J. A. Hicke, P. E.
Higuera, K. M. Mattor, A. W. R. Seddon, H. T. Seppä, J. D. Stednick, and
S. J. Seybold. 2017. Managing bark beetle impacts on ecosystems and
society: Priority questions to motivate future research. Journal of
Applied Ecology 54:750--760.

\hypertarget{ref-negron2009}{}
Negrón, J. F., J. D. McMillin, J. A. Anhold, and D. Coulson. 2009. Bark
beetle-caused mortality in a drought-affected ponderosa pine landscape
in Arizona, USA. Forest Ecology and Management 257:1353--1362.

\hypertarget{ref-north2015}{}
North, M. P., S. L. Stephens, B. M. Collins, J. K. Agee, G. Aplet, J. F.
Franklin, and P. Z. Fule. 2015. Reform forest fire management. Science
349:1280--1281.

\hypertarget{ref-pau2010}{}
Pau, G., F. Fuchs, O. Sklyar, M. Boutros, and W. Huber. 2010. EBImagean
R package for image processing with applications to cellular phenotypes.
Bioinformatics 26:979--981.

\hypertarget{ref-pebesma2019}{}
Pebesma, E., R. Bivand, E. Racine, M. Sumner, I. Cook, T. Keitt, R.
Lovelace, H. Wickham, J. Ooms, K. Müller, and T. L. Pedersen. 2019. Sf:
Simple Features for R.

\hypertarget{ref-plowright2018}{}
Plowright, A. 2018. ForestTools: Analyzing Remotely Sensed Forest Data.

\hypertarget{ref-popescu2004}{}
Popescu, S. C., and R. H. Wynne. 2004. Seeing the Trees in the Forest:
Using Lidar and Multispectral Data Fusion with Local Filtering and
Variable Window Size for Estimating Tree Height. PHOTOGRAMMETRIC
ENGINEERING:16.

\hypertarget{ref-preisler2017}{}
Preisler, H. K., N. E. Grulke, Z. Heath, and S. L. Smith. 2017. Analysis
and out-year forecast of beetle, borer, and drought-induced tree
mortality in California. Forest Ecology and Management. 399: 166-178
399:166--178.

\hypertarget{ref-rcoreteam2018}{}
R Core Team. 2018. R: A Language and Environment for Statistical
Computing. R Foundation for Statistical Computing, Vienna, Austria.

\hypertarget{ref-raffa2008}{}
Raffa, K. F., B. H. Aukema, B. J. Bentz, A. L. Carroll, J. A. Hicke, M.
G. Turner, and W. H. Romme. 2008. Cross-scale Drivers of Natural
Disturbances Prone to Anthropogenic Amplification: The Dynamics of Bark
Beetle Eruptions. BioScience 58:501--517.

\hypertarget{ref-rouse1973}{}
Rouse, W., R. H. Haas, W. Deering, and J. A. Schell. 1973. MONITORING
THE VERNAL ADVANCEMENT AND RETROGRADATION (GREEN WAVE EFFECT) OF NATURAL
VEGETATION. Type II Report, Goddard Space Flight Center, Greenbelt, MD,
USA.

\hypertarget{ref-roussel2019a}{}
Roussel, J.-R. 2019. lidRplugins: Extra functions and algorithms for
lidR package.

\hypertarget{ref-roussel2019}{}
Roussel, J.-R., D. A. (. the documentation), F. D. B. (. bugs and
improved catalog features), and A. S. M. (. lassnags). 2019. lidR:
Airborne LiDAR Data Manipulation and Visualization for Forestry
Applications.

\hypertarget{ref-seidl2015}{}
Seidl, R., J. Müller, T. Hothorn, C. Bässler, M. Heurich, and M. Kautz.
2015. Small beetle, large-scale drivers: How regional and landscape
factors affect outbreaks of the European spruce bark beetle. The Journal
of applied ecology 53:530--540.

\hypertarget{ref-shin2018}{}
Shin, P., T. Sankey, M. Moore, and A. Thode. 2018. Evaluating Unmanned
Aerial Vehicle Images for Estimating Forest Canopy Fuels in a Ponderosa
Pine Stand. Remote Sensing 10:1266.

\hypertarget{ref-stephenson1998}{}
Stephenson, N. 1998. Actual evapotranspiration and deficit: Biologically
meaningful correlates of vegetation distribution across spatial scales.
Journal of Biogeography 25:855--870.

\hypertarget{ref-usdafs2019}{}
USDAFS. 2019, February 11. Press Release: Survey finds 18 million trees
died in California in 2018.
\url{https://www.fs.usda.gov/Internet/FSE_DOCUMENTS/FSEPRD609321.pdf}.

\hypertarget{ref-vega2014}{}
Vega, C., A. Hamrouni, S. El Mokhtari, J. Morel, J. Bock, J. P. Renaud,
M. Bouvier, and S. Durrieu. 2014. PTrees: A point-based approach to
forest tree extraction from lidar data. International Journal of Applied
Earth Observation and Geoinformation 33:98--108.

\hypertarget{ref-young2017}{}
Young, D. J. N., J. T. Stevens, J. M. Earles, J. Moore, A. Ellis, A. L.
Jirka, and A. M. Latimer. 2017. Long-term climate and competition
explain forest mortality patterns under extreme drought. Ecology Letters
20:78--86.

\hypertarget{ref-zhang2016}{}
Zhang, W., J. Qi, P. Wan, H. Wang, D. Xie, X. Wang, and G. Yan. 2016. An
Easy-to-Use Airborne LiDAR Data Filtering Method Based on Cloth
Simulation. Remote Sensing 8:501.


\end{document}

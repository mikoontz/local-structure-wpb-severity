\documentclass[]{article}
\usepackage{lmodern}
\usepackage{amssymb,amsmath}
\usepackage{ifxetex,ifluatex}
\usepackage{fixltx2e} % provides \textsubscript
\ifnum 0\ifxetex 1\fi\ifluatex 1\fi=0 % if pdftex
  \usepackage[T1]{fontenc}
  \usepackage[utf8]{inputenc}
\else % if luatex or xelatex
  \ifxetex
    \usepackage{mathspec}
  \else
    \usepackage{fontspec}
  \fi
  \defaultfontfeatures{Ligatures=TeX,Scale=MatchLowercase}
\fi
% use upquote if available, for straight quotes in verbatim environments
\IfFileExists{upquote.sty}{\usepackage{upquote}}{}
% use microtype if available
\IfFileExists{microtype.sty}{%
\usepackage{microtype}
\UseMicrotypeSet[protrusion]{basicmath} % disable protrusion for tt fonts
}{}
\usepackage[margin=1in]{geometry}
\usepackage{hyperref}
\hypersetup{unicode=true,
            pdfborder={0 0 0},
            breaklinks=true}
\urlstyle{same}  % don't use monospace font for urls
\usepackage{graphicx,grffile}
\makeatletter
\def\maxwidth{\ifdim\Gin@nat@width>\linewidth\linewidth\else\Gin@nat@width\fi}
\def\maxheight{\ifdim\Gin@nat@height>\textheight\textheight\else\Gin@nat@height\fi}
\makeatother
% Scale images if necessary, so that they will not overflow the page
% margins by default, and it is still possible to overwrite the defaults
% using explicit options in \includegraphics[width, height, ...]{}
\setkeys{Gin}{width=\maxwidth,height=\maxheight,keepaspectratio}
\IfFileExists{parskip.sty}{%
\usepackage{parskip}
}{% else
\setlength{\parindent}{0pt}
\setlength{\parskip}{6pt plus 2pt minus 1pt}
}
\setlength{\emergencystretch}{3em}  % prevent overfull lines
\providecommand{\tightlist}{%
  \setlength{\itemsep}{0pt}\setlength{\parskip}{0pt}}
\setcounter{secnumdepth}{0}
% Redefines (sub)paragraphs to behave more like sections
\ifx\paragraph\undefined\else
\let\oldparagraph\paragraph
\renewcommand{\paragraph}[1]{\oldparagraph{#1}\mbox{}}
\fi
\ifx\subparagraph\undefined\else
\let\oldsubparagraph\subparagraph
\renewcommand{\subparagraph}[1]{\oldsubparagraph{#1}\mbox{}}
\fi

%%% Use protect on footnotes to avoid problems with footnotes in titles
\let\rmarkdownfootnote\footnote%
\def\footnote{\protect\rmarkdownfootnote}

%%% Change title format to be more compact
\usepackage{titling}

% Create subtitle command for use in maketitle
\newcommand{\subtitle}[1]{
  \posttitle{
    \begin{center}\large#1\end{center}
    }
}

\setlength{\droptitle}{-2em}

  \title{}
    \pretitle{\vspace{\droptitle}}
  \posttitle{}
    \author{}
    \preauthor{}\postauthor{}
    \date{}
    \predate{}\postdate{}
  
\usepackage[left]{lineno}
\linenumbers
\usepackage{setspace}
\doublespacing

\begin{document}

\section{Relative influence of behavioral and environmental mechanisms
underlying a mass insect-induced tree mortality
event}\label{relative-influence-of-behavioral-and-environmental-mechanisms-underlying-a-mass-insect-induced-tree-mortality-event}

Michael J. Koontz\textsuperscript{1,2,*}, Andrew M.
Latimer\textsuperscript{1,2}, Leif A. Mortenson\textsuperscript{3},
Christopher J. Fettig\textsuperscript{3}, Constance I.
Millar\textsuperscript{4}, Malcolm P. North\textsuperscript{1,2,5}

\textsuperscript{1}Graduate Group in Ecology, University of Californa,
Davis, CA, USA\\
\textsuperscript{2}Department of Plant Sciences, University of
California, Davis, CA, USA\\
\textsuperscript{3}USDA Forest Service, Pacific Southwest Research
Station, Placerville, CA, USA\\
\textsuperscript{4}USDA Forest Service, Pacific Southwest Research
Station, Albany, CA, USA\\
\textsuperscript{5}USDA Forest Service, Pacific Southwest Research
Station, Davis, CA, USA

\textsuperscript{*}Correspondence: \texttt{michael.koontz@colorado.edu}

Date report generated: February 21, 2019

\subsection{Abstract}\label{abstract}

Bark beetles!

\subsection{Introduction}\label{introduction}

Forest spatial structure, the size and distribution of trees in the
forest, is thought to be a key determinant of forest resilience. To
date, much of the work on Sierra Nevada forest resilience focuses on
stem density, which belies the complexity of forest structure and how it
interacts with disturbance. However, complex forest structure is
challenging to quantify, as it requires labor-intensive field surveys
(e.g., to generate stem maps) or highly specialized, expensive equipment
(e.g., LiDAR). Small, unmanned aerial systems (sUAS) enable fast and
relatively cheap remote imaging over dozens of hectares of forest, which
can be used to determine both forest structure and tree condition at the
individual tree scale. Implementing photogrammetry on the collected
images can provide a rich picture of the complex, 3-dimensional forest
structure to which bark beetles respond, and equipping the sUAS with a
multispectral sensor will allow calculation of vegetation indices (e.g.,
NDVI) commonly used to assess tree condition. Latitudinal and
elevational gradients in the intensity of bark beetle activity during
the recent California drought provide unique opportunities for a
postmortem analysis of a major tree die off and how intersecting forces
of forest structure and environmental conditions affect disturbance
dynamics. Quantitative, fine-scale measures of tree condition across
these geographic gradients will enable broad-scale assessment of forest
structure as well as the intensity of western pine beetle-induced tree
mortality. Combined, these measurements can better our understanding of
how complex forest structure affects insect disturbance, and vice versa,
across the Sierra Nevada. Sound forest management requires a better
understanding of the relationships between forest spatial structure,
environmental conditions, and disturbance, which ultimately depends on
accurate measurement of forest structure at appropriate spatial scales.

Aggressive bark beetles dealt the final blow to many of the nearly 150
million trees killed in the California drought of 2012 to 2015 and its
aftermath along a strong south to north latitudinal gradient (Young
\emph{et al.} 2017; USDAFS 2019). A harbinger of climate change effects
to come, high temperatures exacerbating the extreme drought led to tree
mortality events of unprecedented size in the driest, densest forests
across the state (Millar and Stephenson 2015; Young \emph{et al.} 2017).
A century of fire suppression policy has enabled forests to grow
unchecked into dense stands, which increases water stress on trees and
makes them more vulnerable to bark beetle attack (Fettig 2012; North
\emph{et al.} 2015). Previous studies show that bark beetles thrive in
denser forests (Fettig 2012), but density is only a coarse gauge of the
spatial distribution of trees-- the forest structure-- with which bark
beetles interact (Raffa \emph{et al.} 2008). Recent research has shown a
strong link between complex forest structure and forest resilience, but
measuring this complexity generally requires expensive equipment or
labor-intensive field surveys (Larson and Churchill 2012; Kane \emph{et
al.} 2014). These barriers restrict survey frequency and extent, which
limits insights into phenomena like bark beetle outbreaks that rapidly
emerge over weeks to months but have long-lasting effects on forest
conditions. Further, the clear and vast latitudinal gradient of
mortality challenges our ability to simultaneously consider how
environmental conditions may interact with local forest structure to
produce patterns of insect activity.

Forests in California's Sierra Nevada region are characterized by
regular bark beetle disturbances that interact with forest structure.
Bark beetles shape forest structure as they sporadically kill weakened
trees under normal conditions, or wide swaths of even healthy trees
under outbreak conditions. Forest structure also strongly influences
bark beetle activity. Low-density forests are less prone to bark beetle
attacks, but resolving the mechanism underlying this observation
requires a more nuanced view of forest structure. For instance, a
low-density forest may resist attack because its trees are in smaller
clumps with greater average tree vigor, or because its wider canopy
openings disrupt pheromone signaling between beetles (Fettig 2012).
Thus, it remains poorly understood how complex forest structure affects
and is affected by bark beetle activity.

Climate change mitigation strategies emphasize reducing tree densities
(North \emph{et al.} 2015; Young \emph{et al.} 2017), but understanding
the optimal scale and pattern of tree distribution that can mitigate
bark beetle outbreaks will be vital for predicting how California
forests may respond to these interventions. This project investigates
this relationship with the following research questions:

\begin{enumerate}
\def\labelenumi{\arabic{enumi}.}
\item
  At what scale does tree density most strongly correlate with bark
  beetle attack intensity?
\item
  How does local forest structure affect the intensity of bark beetle
  outbreak?
\item
  Are there environmental gradients of elevation or latitude that affect
  bark beetle attack intensity?
\item
  Do these gradients interact with forest structure to shape bark beetle
  attack intensity?
\end{enumerate}

\subsection{Methods}\label{methods}

Study sites were coincident with vegetation plots established by Fettig
\emph{et al.} (2019).

\subsubsection{Study system}\label{study-system}

The study sites comprise mostly ponderosa pine trees, \emph{Pinus
ponderosa}, whose primary bark beetle predator in California is the
western pine beetle (WPB), \emph{Dendroctonus brevicomis}. The WPB is an
aggressive bark beetle, meaning it must attack and kill live trees in
order to successfully reproduce (Raffa \emph{et al.} 2008). Pioneer WPBs
disperse to a new host tree, determine the host's susceptibility to
attack, and use pheromone signals to attract other WPBs. The attracted
WPBs mass attack the tree by boring into its inner bark, laying eggs,
and dying, leaving their offspring to develop inside the doomed tree
before themselves dispersing {[}rRaffa2008{]}. Small WPB populations
prefer weakened trees but large populations can overwhelm the defense
mechanisms of even healthy trees. Successful attacks on large, healthy
trees are boons to bark beetle fecundity and trigger outbreaks in which
populations explode and massive tree mortality occurs. In California,
the WPB can have 3 generations in a single year giving it a greater
potential to spread rapidly through forests than its more infamous
congener, the mountain pine beetle, \emph{Dendroctonus ponderosa} (MPB).

We build our study on 180 vegetation monitoring plots at 36 sites
established between 2016 and 2017. These established plots are located
in beetle-attacked, mixed-conifer forests across the Eldorado,
Stanislaus, Sierra and Sequoia National Forests across an elevation
gradient (3000-4000 feet, 4000-5000 feet, and 5000+ feet above sea
level) and have variable forest structure and disturbance history. Plot
locations were selected specifically in areas with \textgreater{}40\%
ponderosa pine basal area and \textgreater{}10\% ponderosa pine
mortality. The 0.04ha circular plots are clustered along transects in
groups of 5, with between 80 and 200m between each plot. All trees
within the plot were assessed as dead or alive, and the year of death
for dead trees was estimated based on the amount of needles remaining
(no needles= 2+ years prior to survey, very few needles= one year prior
to survey, lots of brown needles = same year as survey). The stem
location of all trees was mapped relative to the center of each plot
using azimuth/distance measurements. Tree identity to species and
diameter at breast height (dbh) were recorded if dbh was greater than
6.35cm.

We will fly preprogrammed transect flights using the sUAS over the 40
hectares surrounding each of transects (with 5 plots each) in the
remaining sites within the Eldorado, Stanislaus, Sierra, and Sequoia
National Forests. As before, two types of data will be collected
simultaneously: 1) georeferenced multispectral reflectance in red, blue,
green, and infrared wavelengths, and 2) georeferenced photographs. Our
primary response variable will be tree condition at the individual tree
scale. We will use the multispectral data to calculate vegetation
indices (e.g., NDVI) for each tree, which can be used to assess the
various stages of bark beetle attack (Näsi \emph{et al.} 2015). The
primary explanatory variable for our study will be the forest structure,
which we can determine from the georeferenced photographs. We will apply
the photogrammetry software to the photographs to stitch them together
and generate a 3-D point cloud (akin to the data returned from LiDAR
instruments) representing complex forest structure. We will develop a
workflow to convert that 3-D point cloud to 2-D forest structure
describing the forest's ICO pattern (Lydersen \emph{et al.} 2013).
Individual trees (I), trees belonging to clusters (C), the number of
trees in each cluster, and the openings between trees (O) will be mapped
(Lydersen \emph{et al.} 2013).

\subsubsection{Statistical model}\label{statistical-model}

The response would still be a Bernoulli distribution of alive/dead We'd
use a logit link to the theta parameter of that Bernoulli distribution
and a linear combination of predictors of logit(theta). That linear
combination of predictors would include: A couple of tree-level
covariates to predict logit(theta): tree size and voronoi polygon area
(need to think more about how this might covary with the kernel
parameters), perhaps topographic position (so as to augment the sub-site
measure of CWD) One sub-site level covariate I'm thinking of to predict
logit(theta): climatic water deficit (there are about 4 CWD pixels per
site; but we could also just assign a bilinearly interpolated CWD value
to each tree to eliminate this sub-site part of the hierarchy) One
site-level effect I'm thinking of: the spatial covariance of
logit(theta) within a site {[}so the site-level parameter to estimate
would be the parameters of the decay kernel at each site{]}

My goal is to tease apart the relative role of environmental drivers
versus behavioral drivers of bark beetle-induced tree mortality. I think
teasing these apart will help with inference about the mechanism
underlying the effect of forest structure on disturbance severity.
Crowded forests means trees are both water stressed and are closer
targets for new attacks {[}i.e., shorter dispersal needed to attack the
next tree{]}, and I think comparing the ``voronoi polygon area'' effect
with the ``spatial covariance of mortality kernel'' effect across sites
will tell us whether it's the water stress or the smaller dispersal
requirements driving mortality patterns. A big voronoi polygon area
effect and a short covariance kernel tells us that it's a water stress
effect-- a crowded tree gets attacked regardless of whether nearby trees
were attacked. A small voronoi polygon area effect and a long covariance
kernel tells us that the mortality is patterned more based on there
being spillover from nearby attacked neighbors instead of how crowded
any given tree is. I expect we might see different relative magnitudes
of voronoi polygon area and covariance kerenel effects depending on CWD.

\subsubsection{Statistical software and data
availability}\label{statistical-software-and-data-availability}

We used \texttt{R} for all statistical analyses (R Core Team 2018). Data
are available via the Open Science Framework.

\subsection{Results}\label{results}

\subsection{Discussion}\label{discussion}

\subsection*{References}\label{references}
\addcontentsline{toc}{subsection}{References}

\hypertarget{refs}{}
\hypertarget{ref-fettig2012b}{}
Fettig CJ. 2012. Chapter 2: Forest health and bark beetles. In: Managing
Sierra Nevada Forests. PSW-GTR-237. USDA Forest Service.

\hypertarget{ref-fettig2019}{}
Fettig CJ, Mortenson LA, Bulaon BM, and Foulk PB. 2019. Tree mortality
following drought in the central and southern Sierra Nevada, California,
U.S. \emph{Forest Ecology and Management} \textbf{432}: 164--78.

\hypertarget{ref-kane2014}{}
Kane VR, North MP, and Lutz JA \emph{et al.} 2014. Assessing fire
effects on forest spatial structure using a fusion of Landsat and
airborne LiDAR data in Yosemite National Park. \emph{Remote Sensing of
Environment} \textbf{151}: 89--101.

\hypertarget{ref-larson2012}{}
Larson AJ and Churchill D. 2012. Tree spatial patterns in fire-frequent
forests of western North America, including mechanisms of pattern
formation and implications for designing fuel reduction and restoration
treatments. \emph{Forest Ecology and Management} \textbf{267}: 74--92.

\hypertarget{ref-lydersen2013}{}
Lydersen JM, North MP, Knapp EE, and Collins BM. 2013. Quantifying
spatial patterns of tree groups and gaps in mixed-conifer forests:
Reference conditions and long-term changes following fire suppression
and logging. \emph{Forest Ecology and Management} \textbf{304}: 370--82.

\hypertarget{ref-millar2015}{}
Millar CI and Stephenson NL. 2015. Temperate forest health in an era of
emerging megadisturbance. \emph{Science} \textbf{349}: 823--6.

\hypertarget{ref-nasi2015}{}
Näsi R, Honkavaara E, and Lyytikäinen-Saarenmaa P \emph{et al.} 2015.
Using UAV-Based Photogrammetry and Hyperspectral Imaging for Mapping
Bark Beetle Damage at Tree-Level. \emph{Remote Sensing} \textbf{7}:
15467--93.

\hypertarget{ref-north2015}{}
North MP, Stephens SL, and Collins BM \emph{et al.} 2015. Reform forest
fire management. \emph{Science} \textbf{349}: 1280--1.

\hypertarget{ref-rcoreteam2018}{}
R Core Team. 2018. R: A Language and Environment for Statistical
Computing. Vienna, Austria: R Foundation for Statistical Computing.

\hypertarget{ref-raffa2008}{}
Raffa KF, Aukema BH, and Bentz BJ \emph{et al.} 2008. Cross-scale
Drivers of Natural Disturbances Prone to Anthropogenic Amplification:
The Dynamics of Bark Beetle Eruptions. \emph{BioScience} \textbf{58}:
501--17.

\hypertarget{ref-usdafs2019}{}
USDAFS. 2019. Press Release: Survey finds 18 million trees died in
California in
2018\url{https://www.fs.usda.gov/Internet/FSE_DOCUMENTS/FSEPRD609321.pdf}.
Viewed 22 Feb 2019.

\hypertarget{ref-young2017}{}
Young DJN, Stevens JT, and Earles JM \emph{et al.} 2017. Long-term
climate and competition explain forest mortality patterns under extreme
drought. \emph{Ecology Letters} \textbf{20}: 78--86.


\end{document}

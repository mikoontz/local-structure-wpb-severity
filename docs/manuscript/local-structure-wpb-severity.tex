\documentclass[]{article}
\usepackage{lmodern}
\usepackage{amssymb,amsmath}
\usepackage{ifxetex,ifluatex}
\usepackage{fixltx2e} % provides \textsubscript
\ifnum 0\ifxetex 1\fi\ifluatex 1\fi=0 % if pdftex
  \usepackage[T1]{fontenc}
  \usepackage[utf8]{inputenc}
\else % if luatex or xelatex
  \ifxetex
    \usepackage{mathspec}
  \else
    \usepackage{fontspec}
  \fi
  \defaultfontfeatures{Ligatures=TeX,Scale=MatchLowercase}
\fi
% use upquote if available, for straight quotes in verbatim environments
\IfFileExists{upquote.sty}{\usepackage{upquote}}{}
% use microtype if available
\IfFileExists{microtype.sty}{%
\usepackage{microtype}
\UseMicrotypeSet[protrusion]{basicmath} % disable protrusion for tt fonts
}{}
\usepackage[margin=1in]{geometry}
\usepackage{hyperref}
\hypersetup{unicode=true,
            pdfborder={0 0 0},
            breaklinks=true}
\urlstyle{same}  % don't use monospace font for urls
\usepackage{graphicx,grffile}
\makeatletter
\def\maxwidth{\ifdim\Gin@nat@width>\linewidth\linewidth\else\Gin@nat@width\fi}
\def\maxheight{\ifdim\Gin@nat@height>\textheight\textheight\else\Gin@nat@height\fi}
\makeatother
% Scale images if necessary, so that they will not overflow the page
% margins by default, and it is still possible to overwrite the defaults
% using explicit options in \includegraphics[width, height, ...]{}
\setkeys{Gin}{width=\maxwidth,height=\maxheight,keepaspectratio}
\IfFileExists{parskip.sty}{%
\usepackage{parskip}
}{% else
\setlength{\parindent}{0pt}
\setlength{\parskip}{6pt plus 2pt minus 1pt}
}
\setlength{\emergencystretch}{3em}  % prevent overfull lines
\providecommand{\tightlist}{%
  \setlength{\itemsep}{0pt}\setlength{\parskip}{0pt}}
\setcounter{secnumdepth}{0}
% Redefines (sub)paragraphs to behave more like sections
\ifx\paragraph\undefined\else
\let\oldparagraph\paragraph
\renewcommand{\paragraph}[1]{\oldparagraph{#1}\mbox{}}
\fi
\ifx\subparagraph\undefined\else
\let\oldsubparagraph\subparagraph
\renewcommand{\subparagraph}[1]{\oldsubparagraph{#1}\mbox{}}
\fi

%%% Use protect on footnotes to avoid problems with footnotes in titles
\let\rmarkdownfootnote\footnote%
\def\footnote{\protect\rmarkdownfootnote}

%%% Change title format to be more compact
\usepackage{titling}

% Create subtitle command for use in maketitle
\newcommand{\subtitle}[1]{
  \posttitle{
    \begin{center}\large#1\end{center}
    }
}

\setlength{\droptitle}{-2em}

  \title{}
    \pretitle{\vspace{\droptitle}}
  \posttitle{}
    \author{}
    \preauthor{}\postauthor{}
    \date{}
    \predate{}\postdate{}
  
\usepackage[left]{lineno}
\linenumbers
\usepackage{setspace}
\doublespacing
\DeclareUnicodeCharacter{200E}{}

\begin{document}

\section{Relative effect of host versus non-host vegetation structure on
forest insect severity depends on climatic water
deficit}\label{relative-effect-of-host-versus-non-host-vegetation-structure-on-forest-insect-severity-depends-on-climatic-water-deficit}

Michael J. Koontz\textsuperscript{1,2,*}, Andrew M.
Latimer\textsuperscript{1,2}, Leif A. Mortenson\textsuperscript{3},
Christopher J. Fettig\textsuperscript{3}, Constance I.
Millar\textsuperscript{4}, Malcolm P. North\textsuperscript{1,2,5}

\textsuperscript{1}Graduate Group in Ecology, University of Californa,
Davis, CA, USA\\
\textsuperscript{2}Department of Plant Sciences, University of
California, Davis, CA, USA\\
\textsuperscript{3}USDA Forest Service, Pacific Southwest Research
Station, Placerville, CA, USA\\
\textsuperscript{4}USDA Forest Service, Pacific Southwest Research
Station, Albany, CA, USA\\
\textsuperscript{5}USDA Forest Service, Pacific Southwest Research
Station, Davis, CA, USA

\textsuperscript{*}Correspondence: \texttt{michael.koontz@colorado.edu}

Date report generated: March 11, 2019

\subsection{Abstract}\label{abstract}

Forest insects are a primary mortality agent of trees in Sierra Nevada
mixed-conifer forests. The recent hot drought from 2012 to 2015 led to
massive tree die-off throughout the state of California, and especially
in the Sierra Nevada.

\subsection{Introduction}\label{introduction}

Aggressive bark beetles dealt the final blow to many of the nearly 150
million trees killed in the California drought of 2012 to 2015 and its
aftermath (USDAFS 2019). A harbinger of climate change effects to come,
high temperatures exacerbating the extreme drought led to tree mortality
events of unprecedented size in the driest, densest forests across the
state (Millar and Stephenson 2015; Young \emph{et al.} 2017). A century
of fire suppression policy has enabled forests to grow unchecked into
dense stands, which increases water stress on trees and makes them more
vulnerable to bark beetle attack (Fettig 2012; North \emph{et al.}
2015).

Previous studies show that bark beetles thrive in denser forests (Fettig
2012), but density is often a coarse gauge of the size and spatial
distribution of trees-- the forest structure-- with which bark beetles
interact (Raffa \emph{et al.} 2008).

Recent research has shown a strong link between complex forest structure
and forest resilience, but measuring this complexity generally requires
expensive equipment or labor-intensive field surveys (Larson and
Churchill 2012; Kane \emph{et al.} 2014). These barriers restrict survey
frequency and extent, which limits insights into phenomena like bark
beetle outbreaks that rapidly emerge over weeks to months but have
long-lasting effects on forest conditions.

Further, the clear and vast latitudinal gradient of mortality challenges
our ability to simultaneously consider how environmental conditions may
interact with local forest structure to produce patterns of insect
activity.

along a strong south to north latitudinal gradient (Young \emph{et al.}
2017; USDAFS 2019). Latitudinal and elevational gradients in the
intensity of bark beetle activity during the recent California drought
provide unique opportunities for a postmortem analysis of a major tree
die off and how intersecting forces of forest structure and
environmental conditions affect disturbance dynamics. Quantitative,
fine-scale measures of tree condition across these geographic gradients
will enable broad-scale assessment of forest structure as well as the
intensity of western pine beetle-induced tree mortality. Combined, these
measurements can better our understanding of how complex forest
structure affects insect disturbance, and vice versa, across the Sierra
Nevada. Sound forest management requires a better understanding of the
relationships between forest spatial structure, environmental
conditions, and disturbance, which ultimately depends on accurate
measurement of forest structure at appropriate spatial scales.

Forests in California's Sierra Nevada region are characterized by
regular bark beetle disturbances that interact with forest structure.
Bark beetles shape forest structure as they sporadically kill weakened
trees under normal conditions, or wide swaths of even healthy trees
under outbreak conditions. Forest structure also strongly influences
bark beetle activity. Low-density forests are less prone to bark beetle
attacks, but resolving the mechanism underlying this observation
requires a more nuanced view of forest structure. For instance, a
low-density forest may resist attack because its trees are in smaller
clumps with greater average tree vigor, or because its wider canopy
openings disrupt pheromone signaling between beetles (Fettig 2012).
Thus, it remains poorly understood how complex forest structure affects
and is affected by bark beetle activity.

Forest spatial structure, the size and distribution of trees in the
forest, is thought to be a key determinant of forest resilience. To
date, much of the work on Sierra Nevada forest resilience focuses on
stem density, which belies the complexity of forest structure and how it
interacts with disturbance. However, complex forest structure is
challenging to quantify, as it requires labor-intensive field surveys
(e.g., to generate stem maps) or highly specialized, expensive equipment
(e.g., LiDAR). Small, unmanned aerial systems (sUAS) enable fast and
relatively cheap remote imaging over dozens of hectares of forest, which
can be used to determine both forest structure and tree condition at the
individual tree scale. Implementing photogrammetry on the collected
images can provide a rich picture of the complex, 3-dimensional forest
structure to which bark beetles respond, and equipping the sUAS with a
multispectral sensor will allow calculation of vegetation indices (e.g.,
NDVI) commonly used to assess tree condition.

Climate change mitigation strategies emphasize reducing tree densities
(North \emph{et al.} 2015; Young \emph{et al.} 2017), but understanding
the optimal scale and pattern of tree distribution that can mitigate
bark beetle outbreaks will be vital for predicting how California
forests may respond to these interventions. This project investigates
this relationship with the following research questions:

\begin{enumerate}
\def\labelenumi{\arabic{enumi}.}
\item
  How does local host tree density affect the severity of western pine
  beetle disturbance?
\item
  How does total tree density affect the severity of western pine beetle
  disturbance?
\item
  How does environmentally-driven tree moisture stress affect the
  severity of western pine beetle disturbance?
\item
  Do the effects of forest structure and environmental condition on
  western pine beetle disturbance interact?
\end{enumerate}

We used ultra-high resolution remote sensing data from a small,
unhumanned aerial system (sUAS, aka drone) over a network of 32 sites in
the Sierra Nevada spanning 1000m of elevation and 350km of latitude and
covering a total of 9 square kilometers of forest to ask how fine-scale
forest structure affected the probability of tree mortality during the
cumulative mortality event of

\subsection{Methods}\label{methods}

\subsubsection{Study system}\label{study-system}

The study sites comprise mostly ponderosa pine trees, \emph{Pinus
ponderosa}, whose primary bark beetle predator in California is the
western pine beetle (WPB), \emph{Dendroctonus brevicomis}. The WPB is an
aggressive bark beetle, meaning it must attack and kill live trees in
order to successfully reproduce (Raffa \emph{et al.} 2008). Pioneer WPBs
disperse to a new host tree, determine the host's susceptibility to
attack, and use pheromone signals to attract other WPBs. The attracted
WPBs mass attack the tree by boring into its inner bark, laying eggs,
and dying, leaving their offspring to develop inside the doomed tree
before themselves dispersing (Raffa \emph{et al.} 2008). Small WPB
populations prefer weakened trees but large populations can overwhelm
the defense mechanisms of even healthy trees. Successful attacks on
large, healthy trees are boons to bark beetle fecundity and trigger
outbreaks in which populations explode and massive tree mortality
occurs. In California, the WPB can have 3 generations in a single year
giving it a greater potential to spread rapidly through forests than its
more infamous congener, the mountain pine beetle, \emph{Dendroctonus
ponderosa} (MPB).

We built our study on 180 vegetation monitoring plots at 36 sites
established between 2016 and 2017 (Fettig \emph{et al.} 2019). These
established plots are located in beetle-attacked, mixed-conifer forests
across the Eldorado, Stanislaus, Sierra and Sequoia National Forests
across an elevation gradient (3000-4000 feet, 4000-5000 feet, and 5000+
feet above sea level) and have variable forest structure and disturbance
history. Plot locations were selected specifically in areas with
\textgreater{}40\% ponderosa pine basal area and \textgreater{}10\%
ponderosa pine mortality. The 0.04ha circular plots are clustered along
transects in groups of 5, with between 80 and 200m between each plot.
All trees within the plot were assessed as dead or alive. The stem
location of all trees was mapped relative to the center of each plot
using azimuth/distance measurements. Tree identity to species and
diameter at breast height (dbh) were recorded if dbh was greater than
6.35cm. During the spring and early summer of 2018, all field plots were
revisited to assess whether dead trees had fallen.

\subsubsection{Instrumentation}\label{instrumentation}

Imagery was captured using a DJI Zenmuse X3 RGB camera (DJI 2015a) and a
Micasense RedEdge3 5-band multispectral camera (Micasense 2015). We
mounted both of these instruments simultaneously on a DJI Matrice 100
aircraft (DJI 2015b) using the DJI 3-axis stabilized gimbal for the
Zenmuse X3 camera and a Micasense angled fixed mount for the RedEdge3
camera. The gimbal and the angled fixed mount ensured both instruments
were nadir-facing during image capture. Just prior or after image
capture at each site, we calibrated the RedEdge3 camera by taking an
image of a calibration panel on the ground in full sun with known
reflectance values for each of the 5 narrow bands.

Table XXXXX. Reflectance sensitivity of the Micasense Rededge3 camera.
+------------+--------------+------------------+------------------+----------+-----------------+---------------+
\textbar{} Band~ \textbar{} Band~ \textbar{} Band~ \textbar{} Center~
\textbar{} Band~ \textbar{} Wavelength~ \textbar{} Calibration~
\textbar{} \textbar{} number \textbar{} name \textbar{} abbreviation
\textbar{} wavelength \textbar{} width \textbar{} range \textbar{} panel
value \textbar{}
+============+==============+==================+==================+==========+=================+===============+
\textbar{} 1 \textbar{}blue \textbar{}b \textbar{} 475 \textbar{} 20
\textbar{} 465-485 \textbar{} 0.63 \textbar{}
+------------+--------------+------------------+------------------+----------+-----------------+---------------+
\textbar{} 2 \textbar{}green \textbar{}g \textbar{} 560 \textbar{} 20
\textbar{} 550-570 \textbar{} 0.63 \textbar{}
+------------+--------------+------------------+------------------+----------+-----------------+---------------+
\textbar{} 3 \textbar{}red \textbar{}r \textbar{} 668 \textbar{} 10
\textbar{} 663-673 \textbar{} 0.63 \textbar{}
+------------+--------------+------------------+------------------+----------+-----------------+---------------+
\textbar{} 4 \textbar{}near infrared \textbar{}nir \textbar{} 840
\textbar{} 40 \textbar{} 820-860 \textbar{} 0.64 \textbar{}
+------------+--------------+------------------+------------------+----------+-----------------+---------------+
\textbar{} 5 \textbar{}red edge \textbar{}re \textbar{} 717 \textbar{}
10 \textbar{} 712-722 \textbar{} 0.60 \textbar{}
+------------+--------------+------------------+------------------+----------+-----------------+---------------+

\subsubsection{Flight protocol}\label{flight-protocol}

Image capture was conducted as close to solar noon as possible to
minimize shadow effects (always within 4 hours; usually within 2 hours).
Prior to the aerial survey, two strips of bright orange drop cloth
(\textasciitilde{}100cm x 15cm) were positioned as an ``X'' over the
permanent monuments marking the center of the 5 field plots from Fettig
\emph{et al.} (2019).

For each of the 36 sites (containing 5 plots each), we captured imagery
over the surrounding \textasciitilde{}40 hectares of forested area using
north-south aerial transects. For XXXXX sites, we surveyed less
surrounding area in order to maintain visual and radio communication
with the aircraft during flight (Table XXXXXX).

We preprogrammed transect paths using Map Pilot for DJI on iOS
(hereafter Map Pilot) (Easy 2018). All transects tracked the terrain and
their altitude remained approximately constant at 120 meters above
ground level in order to maintain consistent ground sampling distance in
the imagery. Ground level was based on a 1-arc-second digital elevation
model (Farr \emph{et al.} 2007) and we implemented terrain following
using Map Pilot. For this analysis, we dropped 4 sites whose imagery was
of insufficient quality to process.

Structure from motion (SfM) processing requires highly overlapping
images, especially in densely vegetated areas. We planned transects with
90\% forward overlap and 90\% side overlap at 100 meters below the lens.
Thus, with flights being at 120 meters above ground level, we achieved
slightly higher than 90/90 overlap for objects 20 meters tall or
shorter. Overlap values were based on focal length and field of view
parameters of the Zenmuse X3 camera. Images were captured at a constant
rate of 1 image every 2 seconds for both cameras. A forward overlap of
90\% at 100 meters translates to a flight speed of approximately 6.3 m/s
and a side overlap of 90\% at 100 meters translates to transects
approximately 18 meters apart. Approximately 1900 photos were captured
over each 40 hectare survey area for each camera.

\subsubsection{Structure from motion/Photogrammetric
processing}\label{structure-from-motionphotogrammetric-processing}

We used structure from motion (SfM), aka photogrammetry, to generate
orthorectified reflectance maps, digital surface models, and dense point
clouds for each field site. We used Pix4Dmapper Cloud to process imagery
using parameters ideal for images of a densely vegetated area taken by a
multispectral camera. For three sites, we processed the RGB and the
multispectral imagery in the same project to enhance the resolution of
the dense point cloud. All SfM projects resulted in a single processing
``block,'' indicating that all images in the project were optimized and
processed together.

\subsubsection{Creating canopy height
models}\label{creating-canopy-height-models}

We classified each survey area's dense point cloud into ``ground'' and
``non-ground'' points using a cloth simulation filter algorithm (Zhang
\emph{et al.} 2016) implemented in the \texttt{lidR} (Roussel \emph{et
al.} 2019) package. We rasterized the ground points using the
\texttt{raster} package (Hijmans \emph{et al.} 2019) to create a digital
terrain model representing the ground underneath the vegetation at 1
meter resolution. We created a canopy height model by subtracting the
digital terrain model from the digital surface model created in
Pix4Dmapper.

\subsubsection{Tree detection}\label{tree-detection}

We tested a total of 7 automatic tree detection algorithms and a total
of 177 parameter sets on the canopy height model or the dense point
cloud to locate trees within each site (Table XXXXX; algorithm, number
of parameter sets, reference). We used 3 parameter sets of a variable
window filter implmented in \texttt{ForestTools} (Plowright 2018)
including the default variable window filter function in
\texttt{ForestTools} as well as the ``pines'' and ``combined'' functions
from Popescu and Wynne (2004). We used 6 parameter sets of a local
maximum filter implemented in \texttt{lidR}. We used 131 parameter sets
of the algorithm from Li \emph{et al.} (2012), which operates on the
original point cloud. These parameter sets included those from Shin
\emph{et al.} (2018) and Jakubowski \emph{et al.} (2013). We used 3
parameter sets of the \texttt{watershed} algorithm implemented in
\texttt{lidR}, which is a wrapper for a function in the \texttt{EBImage}
package (Pau \emph{et al.} 2010). We used 3 parameter sets of
\texttt{ptrees} (Vega \emph{et al.} 2014) implemented in \texttt{lidR}
(Roussel \emph{et al.} 2019) and \texttt{lidRplugins} (Roussel 2019) and
which operates on the raw point cloud, without first normalizing it to
height above ground level (i.e.. subtracting the ground elevation from
the dense point cloud). We used the default parameter set of the
\texttt{multichm} (Eysn \emph{et al.} 2015) algorithm implmented in
\texttt{lidR} (Roussel \emph{et al.} 2019) and \texttt{lidRplugins}
(Roussel 2019). We used 30 parameter sets of the experimental algorithm
\texttt{lmfx} (Roussel 2019).

\subsubsection{Map ground data}\label{map-ground-data}

Each orthorectified reflectance map was inspected to locate the 5 orange
``X''s marking the center of the field plots. We were able to locate 110
out of 180 field plots and were then able to use these plots for
validation of automated tree detection algorithms. We used the
\texttt{sf} package (Pebesma \emph{et al.} 2019) to convert
distance-from-center and azimuth measurements of each tree in the ground
plots to an x-y position on the SfM-derived reflectance map using the
x-y position of the orange X visible in the reflectance map as the
center.

\subsubsection{Correspondence of automatic tree detection with ground
data}\label{correspondence-of-automatic-tree-detection-with-ground-data}

We calculated 7 forest structure metrics for each field plot using the
ground data collected by Fettig \emph{et al.} (2019): total number of
trees, number of trees greater than 15 meters, number of trees less than
15 meters, mean height of trees, 25\textsuperscript{th} percentile tree
height, 75\textsuperscript{th} percentile tree height, mean distance to
nearest tree neighbor, mean distance to 2\textsuperscript{nd} nearest
neighbor.

For each tree detection algorithm and parameter set described above, we
calculated the same set of 7 structure metrics within the footprint of
the validation field plots. We calculated the Pearson's correlation and
root mean square error (RMSE) between the ground data and the aerial
data for each of the 7 structure metrics for each of the XXXXX automatic
tree detection algorithms.

For each algorithm and parameter set, we calculated its performance
relative to other algorithms as whether its Pearson's correlation was
within 5\% of the highest Pearson's correlation as well as whether its
RMSE was within 5\% of the lowest RMSE. For each algorithm/parameter
set, we summed the number of forest structure metrics for which it
reached these 5\% thresholds. For automatically detecting trees across
the whole study, we selected the algorithm/parameter set that performed
well across the most number of forest metrics.

\subsubsection{Segmentation of crowns}\label{segmentation-of-crowns}

We delineated individual tree crowns with a marker controlled watershed
segmentation algorithm (Meyer and Beucher 1990) using the detected
treetops as markers implemented in the \texttt{ForestTools} package
(Plowright 2018). If the automatic segmentation algorithm failed to
generate a crown segment for a detected tree (e.g., often snags with a
very small crown footprint), a circular crown was generated with a
radius of 0.5 meters. If the segmentation generated multiple polygons
for a single detected tree, only the polygon containing the detected
tree was retained. Image overlap decreases near the edges of the overall
flight path, which reduces the quality of the SfM processing in those
areas. Thus, we excluded segmented crowns within 35 meters of the edge
of the survey area.

We used the \texttt{velox} package (Hunziker 2017) to extract all the
pixel values from the orthorectified reflectance map for each of the 5
narrow bands within each segmented crown polygon. Per pixel, we
additionally calculated the normalized difference vegetation index
(NDVI; Rouse \emph{et al.} (1973)), the normalized difference red edge
(NDRE; Gitelson and Merzlyak (1994)), the red-green index (RGI; Coops
\emph{et al.} (2006)), the red edge chlorophyll index (CI{[}red edge{]};
Clevers and Gitelson (2013)), and the green chlorophyll index
(CI{[}green{]}; Clevers and Gitelson (2013)). For each crown polygon, we
calculated the mean value for each raw and derived reflectance band (5
raw; 5 derived).

\subsubsection{Classification of trees}\label{classification-of-trees}

We overlaid the segmented crowns on the reflectance maps from 20 sites
spanning the latitudinal and elevational gradient in the study. Using
QGIS, we hand classified XXXX trees as live/dead and as one of 5
dominant species in the study area (\emph{Pinus ponderosa}, \emph{Pinus
lambertiana}, \emph{Abies concolor}, \emph{Calocedrus decurrens}, or
\emph{Quercus kelloggi}) using the mapped ground data as a guide.

We used all 10 mean values of the reflectance bands for each tree crown
polygon to predict whether the hand classified trees were alive or dead
using a boosted logistic regression model implemented in the
\texttt{caret} package (Kuhn 2008). For just the living trees, we
similarly used all 10 reflectance values to predict the tree species
using regularized discriminant analysis implemented in the
\texttt{caret} package, which proved to have the highest accuracy for a
training dataset (accuracy = XXXXX, kappa = XXXXX).

Finally, we used these models to classify all tree crowns in the data
set as alive or dead as well as the species of living trees.

\subsubsection{Allometric scaling of height to basal
area}\label{allometric-scaling-of-height-to-basal-area}

We converted the height of each tree (known from the canopy height
model) to its basal area. Using the tree height and diameter at breast
height (DBH; breast height = 1.37m) ground data, we fit a simple linear
regression to predict DBH from height for each of the 5 dominant
species. Using the model-classified tree species of each segmented tree,
we used the corresponding linear relationship for that species to
estimate the DBH given the tree's height. We then calculated each tree's
basal area, assuming no tapering from breast height.

\subsubsection{Note on assumptions about dead
trees}\label{note-on-assumptions-about-dead-trees}

For the purposes of this study, we assumed that all dead trees were
ponderosa pine. This is a reasonably good assumption, given that Fettig
\emph{et al.} (2019) found that XXXXX\% (\textasciitilde{}90) of the
dead trees in the coincident ground plots were ponderosa pine.

\subsubsection{Rasterizing individual tree
data}\label{rasterizing-individual-tree-data}

Because the tree detection algorithms were validated against ground data
at the plot level, we rasterized the classified trees at a spatial
resolution similar to that of the ground plots (rasterized to 20m x 20m
equalling 400 m\textsuperscript{2}; circular ground plots with 11.35m
radius equalling 404 m\textsuperscript{2}). In each raster cell, we
tallied: number of alive trees, number of dead trees, number of
ponderosa pine trees, number of non-ponderosa pine trees, basal area of
ponderosa pine trees, basal area of non-ponderosa pine trees.

\subsubsection{Environmental data}\label{environmental-data}

We used climatic water deficit (CWD) (Stephenson 1998) from the
1980-2010 mean value of the basin characterization model (Flint \emph{et
al.} 2013) as an integrated measure of temperature and moisture
conditions for each cell. Higher values of CWD correspond to hotter,
drier conditions and lower values correspond to cooler, wetter
conditions CWD has been shown to correlate well with broad patterns of
tree mortality in the Sierra Nevada (Young \emph{et al.} 2017). We
resampled the climatic water deficit product using bilinear
interpolation implemented in the \texttt{raster} package to match the
20m x 20m spatial scale of the other variables. We converted the CWD
value for each cell into a z-score representing that cell's deviation
from the mean CWD across the climatic range of Sierra Nevada ponderosa
pine as determined from XXXXX herbarium records described in Baldwin
\emph{et al.} (2017). Thus, a CWD z-score of one would indicate that the
CWD at that cell is one standard deviation hotter/drier than the mean
CWD across all geolocated herbarium records for ponderosa pine in the
Sierra Nevada.

\subsubsection{Statistical model}\label{statistical-model}

We used a generalized additive model with a binomial response and a
logit link to predict the probability of ponderosa pine mortality within
each raster cell as a function of the crossed effects of ponderosa pine
quadratic mean diameter and count added to the crossed effect of
non-ponderosa pine quadratic mean diameter and count as well as the
interaction of each summand with climatic water deficit. To account for
spatial autocorrelation of the processes underlying ponderosa mortality,
we included a separate smoothing term per site of the interaction
between the x- and y-position of each cell using the \texttt{mgcv}
package (Wood \emph{et al.} 2016) with a thin plate spline basis having
a dimension of 10.

Model checking\ldots{}

\[
\begin{aligned}
y_i \sim &\ Binom(n_i, \theta_i) \\
logit(\theta_i) = &\ \beta_0 + \gamma_i\ + \\
& \beta_1X_{cwd, i}\ + \\
& \beta_1X_{cwd, i}(\beta_2X_{pipoBA, i} + \beta_3X_{pipoCount, i} + \beta_4X_{pipoBA, i}X_{pipoCount, i})\ + \\ 
& \beta_1X_{cwd, i}(\beta_5X_{nonpipoBA, i} + \beta_6X_{nonpipoCount, i} + \beta_7X_{nonpipoBA, i}X_{nonpipoCount, i}) \\
\gamma_i \sim &\ Norm(\mu = 0, \sigma^2)
\end{aligned}
\] Where \(y_i\) is the number of dead ponderosa pine trees in cell
\(i\), \(n_i\) is the total number of ponderosa pine trees in cell
\(i\), \(\theta_i\) is the probability of a pine tree dying in cell
\(i\), \(X_{cwd, i}\) is the z-score of climatic water deficit,
\(X_{pipoBA, i}\) is the centered basal area of ponderosa pine in a
cell, \(X_{pipoCount, i}\) is the centered number of ponderosa pine
trees in a cell, \(X_{nonpipoBA, i}\) is the centered total basal area
of non-ponderosa pine trees in a cell, \(X_{nonpipoCount, i}\) is the
centered number of non-ponderosa pine trees in a cell, and \(\gamma_i\)
is a random deviation from the log-odds of ponderosa pine mortality at
mean values of ponderosa pine basal area and count, mean values of
non-ponderosa pine basal area and count, and mean value of climatic
water deficit across the full range of Sierra Nevada ponderosa pine.
\(\sigma^2\) represents the pooled variance of these random deviations.

\subsubsection{Software and data
availability}\label{software-and-data-availability}

All data are available via the Open Science Framework. Statistical
analyses were performed using the \texttt{mgcv} and \texttt{brms}
packages. With the exception of the SfM software (Pix4Dmapper Cloud) and
the GIS software QGIS, all data carpentry and analyses were performed
using \texttt{R} (R Core Team 2018).

\subsection{Results}\label{results}

\subsubsection{Tree detection}\label{tree-detection-1}

We found that the experimental \texttt{lmfx} algorithm with parameter
values of XXXXX (Roussel \emph{et al.} 2019) performed the best across 7
measures of forest structure as measured by Pearson's correlation with
ground data (Table XXXX; rows are different forest metrics, columns are
correlation and RMSE).

\subsubsection{Effect of local structure on western pine beetle
severity}\label{effect-of-local-structure-on-western-pine-beetle-severity}

We found a strong main effect of climatic water deficit on the
probability of ponderosa pine mortality within each 20m x 20m cell.
Greater climatic water deficit, indicating hotter/drier conditions,
increased the probability of ponderosa pine mortality.

We also found a strong effect of ponderosa pine basal area, accounting
for the total basal area with greater ponderosa pine basal area
increasing the probability of ponderosa pine mortality.

We found a negative effect of total basal area on the probability of
ponderosa pine mortality.

We found no 2-way interaction between ponderosa pine basal area and
total basal area.

We found a significant 3-way interaction between ponderosa pine basal
area, total basal area, and climatic water deficit. In hotter, drier
sites, a positive interaction between ponderosa pine basal area and
total basal area emerges.

\subsection{Discussion}\label{discussion}

\subsubsection{\texorpdfstring{Similarities and differences with Fettig
\emph{et al.}
(2019)}{Similarities and differences with Fettig et al. (2019)}}\label{similarities-and-differences-with-fettig2019}

Fettig \emph{et al.} (2019) found positive relationship between number
of trees killed and: total number of trees, total basal area, stand
density index.

Fettig \emph{et al.} (2019) found negative relationship between the
proportion of trees killed and: total number of trees, stand density
index.

Hayes \emph{et al.} (2009) and Fettig \emph{et al.} (2019) found
measures of host availability explained less variation in mortality than
measures of stand density.

Negrón \emph{et al.} (2009) reported positive association of probability
of ponderosa pine mortality and tree density during a drought in
Arizona.

Effect of competition may be masked because drought was so extreme
Fettig \emph{et al.} (2019); Floyd \emph{et al.} (2009), which is
perhaps why we saw a counter-intuitive signal of increasing total basal
area leading to lower probability of ponderosa pine mortality.

\subsubsection{Broader context around field
plots}\label{broader-context-around-field-plots}

We surveyed 9 square kilometers of forest representing XXXXXX trees
along a broad gradient. Site selection and small plot size can influence
inference. For instance, Fettig \emph{et al.} (2019) reported
statistically undetectable differences in overall mortality in their
plot network across 4 national forests. By expanding the hectarage
surveyed by a factor of 200, we detected dramatic differences in overall
mortality.

This is about more than sample size. This is also about capturing the
local disturbance phenomenon.

\subsubsection{Closer spacing between potential host trees facilitates
dispersal}\label{closer-spacing-between-potential-host-trees-facilitates-dispersal}

If this drives mortality patterns, then we'd expect the count of
ponderosa pine trees, accounting for other variables, to have a strong
positive effect.

\subsubsection{Host preference for large
trees}\label{host-preference-for-large-trees}

If this drives mortality patterns, then we'd expect the quadratic mean
diameter of ponderosa pine trees, accounting for other variables, to
have a strong positive effect.

\subsubsection{Denser forests augment pheromone
communication}\label{denser-forests-augment-pheromone-communication}

If this drives mortality patterns, then we'd expect the count of all
trees, accounting for other variables, to have a strong positive effect.

\subsubsection{Tree crowding leads to greater average water stress per
tree}\label{tree-crowding-leads-to-greater-average-water-stress-per-tree}

If this drives mortality patterns, then we'd expect the quadratic mean
dimater of all trees, accounting for other factors, to have a strong
positive effect.

\subsubsection{Interaction between host density and host
size}\label{interaction-between-host-density-and-host-size}

A positive coefficient would indicate a combined effect of WPB
preference for large trees and nearby host availability.

\subsubsection{Interaction between all tree density and all tree
size}\label{interaction-between-all-tree-density-and-all-tree-size}

A positive coefficient would indicate a combined effect of tree crowding
and pheromone communication enhancement.

\subsubsection{Interactions with climatic water
deficit}\label{interactions-with-climatic-water-deficit}

Are any of the above mechanisms exacerbated by water stress of the
trees?

\subsubsection{Important considerations}\label{important-considerations}

Cumulative effect of elevated insect activity, as mortality was spread
out over 5 years and we surveyed at the end.

\subsubsection{Future directions}\label{future-directions}

My goal is to tease apart the relative role of environmental drivers
versus behavioral drivers of bark beetle-induced tree mortality. I think
teasing these apart will help with inference about the mechanism
underlying the effect of forest structure on disturbance severity.
Crowded forests means trees are both water stressed and are closer
targets for new attacks {[}i.e., shorter dispersal needed to attack the
next tree{]}, and I think comparing the ``voronoi polygon area'' effect
with the ``spatial covariance of mortality kernel'' effect across sites
will tell us whether it's the water stress or the smaller dispersal
requirements driving mortality patterns. A big voronoi polygon area
effect and a short covariance kernel tells us that it's a water stress
effect-- a crowded tree gets attacked regardless of whether nearby trees
were attacked. A small voronoi polygon area effect and a long covariance
kernel tells us that the mortality is patterned more based on there
being spillover from nearby attacked neighbors instead of how crowded
any given tree is. I expect we might see different relative magnitudes
of voronoi polygon area and covariance kerenel effects depending on CWD.

\subsection*{References}\label{references}
\addcontentsline{toc}{subsection}{References}

\hypertarget{refs}{}
\hypertarget{ref-baldwin2017a}{}
Baldwin BG, Thornhill AH, and Freyman WA \emph{et al.} 2017. Species
richness and endemism in the native flora of California. \emph{American
Journal of Botany} \textbf{104}: 487--501.

\hypertarget{ref-clevers2013}{}
Clevers J and Gitelson A. 2013. Remote estimation of crop and grass
chlorophyll and nitrogen content using red-edge bands on Sentinel-2 and
-3. \emph{International Journal of Applied Earth Observation and
Geoinformation} \textbf{23}: 344--51.

\hypertarget{ref-coops2006}{}
Coops NC, Johnson M, Wulder MA, and White JC. 2006. Assessment of
QuickBird high spatial resolution imagery to detect red attack damage
due to mountain pine beetle infestation. \emph{Remote Sensing of
Environment} \textbf{103}: 67--80.

\hypertarget{ref-dji2015}{}
DJI. 2015a. Zenmuse X3 - Creativity
Unleashed\url{https://www.dji.com/zenmuse-x3/info}. Viewed 4 Mar 2019.

\hypertarget{ref-dji2015a}{}
DJI. 2015b. DJI - The World Leader in Camera Drones/Quadcopters for
Aerial Photography\url{https://www.dji.com/matrice100/info}. Viewed 4
Mar 2019.

\hypertarget{ref-dronesmadeeasy2018}{}
Easy DM. 2018. ‎Map Pilot for
DJI\url{https://itunes.apple.com/us/app/map-pilot-for-dji/id1014765000?mt=8}.
Viewed 4 Mar 2019.

\hypertarget{ref-eysn2015}{}
Eysn L, Hollaus M, and Lindberg E \emph{et al.} 2015. A Benchmark of
Lidar-Based Single Tree Detection Methods Using Heterogeneous Forest
Data from the Alpine Space. \emph{Forests} \textbf{6}: 1721--47.

\hypertarget{ref-farr2007}{}
Farr TG, Rosen PA, and Caro E \emph{et al.} 2007. The Shuttle Radar
Topography Mission. \emph{Reviews of Geophysics} \textbf{45}.

\hypertarget{ref-fettig2012b}{}
Fettig CJ. 2012. Chapter 2: Forest health and bark beetles. In: Managing
Sierra Nevada Forests. PSW-GTR-237. USDA Forest Service.

\hypertarget{ref-fettig2019}{}
Fettig CJ, Mortenson LA, Bulaon BM, and Foulk PB. 2019. Tree mortality
following drought in the central and southern Sierra Nevada, California,
U.S. \emph{Forest Ecology and Management} \textbf{432}: 164--78.

\hypertarget{ref-flint2013}{}
Flint LE, Flint AL, Thorne JH, and Boynton R. 2013. Fine-scale
hydrologic modeling for regional landscape applications: The California
Basin Characterization Model development and performance.
\emph{Ecological Processes} \textbf{2}: 25.

\hypertarget{ref-floyd2009}{}
Floyd ML, Clifford M, and Cobb NS \emph{et al.} 2009. Relationship of
stand characteristics to drought-induced mortality in three Southwestern
piñonJuniper woodlands. \emph{Ecological Applications} \textbf{19}:
1223--30.

\hypertarget{ref-gitelson1994}{}
Gitelson A and Merzlyak MN. 1994. Spectral Reflectance Changes
Associated with Autumn Senescence of Aesculus hippocastanum L. and Acer
platanoides L. Leaves. Spectral Features and Relation to Chlorophyll
Estimation. \emph{Journal of Plant Physiology} \textbf{143}: 286--92.

\hypertarget{ref-hayes2009}{}
Hayes CJ, Fettig CJ, and Merrill LD. 2009. Evaluation of Multiple Funnel
Traps and Stand Characteristics for Estimating Western Pine
Beetle-Caused Tree Mortality. \emph{Journal of Economic Entomology}
\textbf{102}: 2170--82.

\hypertarget{ref-hijmans2019}{}
Hijmans RJ, Etten J van, and Sumner M \emph{et al.} 2019. Raster:
Geographic Data Analysis and Modeling.

\hypertarget{ref-hunziker2017}{}
Hunziker P. 2017. Velox: Fast Raster Manipulation and Extraction.

\hypertarget{ref-jakubowski2013}{}
Jakubowski MK, Li W, Guo Q, and Kelly M. 2013. Delineating Individual
Trees from Lidar Data: A Comparison of Vector- and Raster-based
Segmentation Approaches. \emph{Remote Sensing} \textbf{5}: 4163--86.

\hypertarget{ref-kane2014}{}
Kane VR, North MP, and Lutz JA \emph{et al.} 2014. Assessing fire
effects on forest spatial structure using a fusion of Landsat and
airborne LiDAR data in Yosemite National Park. \emph{Remote Sensing of
Environment} \textbf{151}: 89--101.

\hypertarget{ref-kuhn2008}{}
Kuhn M. 2008. Building Predictive Models in R Using the caret Package.
\emph{Journal of Statistical Software} \textbf{28}: 1--26.

\hypertarget{ref-larson2012}{}
Larson AJ and Churchill D. 2012. Tree spatial patterns in fire-frequent
forests of western North America, including mechanisms of pattern
formation and implications for designing fuel reduction and restoration
treatments. \emph{Forest Ecology and Management} \textbf{267}: 74--92.

\hypertarget{ref-li2012}{}
Li W, Guo Q, Jakubowski MK, and Kelly M. 2012. A New Method for
Segmenting Individual Trees from the Lidar Point Cloud.
\emph{Photogrammetric Engineering \& Remote Sensing} \textbf{78}:
75--84.

\hypertarget{ref-meyer1990}{}
Meyer F and Beucher S. 1990. Morphological segmentation. \emph{Journal
of Visual Communication and Image Representation} \textbf{1}: 21--46.

\hypertarget{ref-micasense2015}{}
Micasense. 2015.
MicaSense\url{https://support.micasense.com/hc/en-us/articles/215261448-RedEdge-User-Manual-PDF-Download-}.
Viewed 4 Mar 2019.

\hypertarget{ref-millar2015}{}
Millar CI and Stephenson NL. 2015. Temperate forest health in an era of
emerging megadisturbance. \emph{Science} \textbf{349}: 823--6.

\hypertarget{ref-negron2009}{}
Negrón JF, McMillin JD, Anhold JA, and Coulson D. 2009. Bark
beetle-caused mortality in a drought-affected ponderosa pine landscape
in Arizona, USA. \emph{Forest Ecology and Management} \textbf{257}:
1353--62.

\hypertarget{ref-north2015}{}
North MP, Stephens SL, and Collins BM \emph{et al.} 2015. Reform forest
fire management. \emph{Science} \textbf{349}: 1280--1.

\hypertarget{ref-pau2010}{}
Pau G, Fuchs F, and Sklyar O \emph{et al.} 2010. EBImagean R package for
image processing with applications to cellular phenotypes.
\emph{Bioinformatics} \textbf{26}: 979--81.

\hypertarget{ref-pebesma2019}{}
Pebesma E, Bivand R, and Racine E \emph{et al.} 2019. Sf: Simple
Features for R.

\hypertarget{ref-plowright2018}{}
Plowright A. 2018. ForestTools: Analyzing Remotely Sensed Forest Data.

\hypertarget{ref-popescu2004}{}
Popescu SC and Wynne RH. 2004. Seeing the Trees in the Forest: Using
Lidar and Multispectral Data Fusion with Local Filtering and Variable
Window Size for Estimating Tree Height. \emph{PHOTOGRAMMETRIC
ENGINEERING}: 16.

\hypertarget{ref-rcoreteam2018}{}
R Core Team. 2018. R: A Language and Environment for Statistical
Computing. Vienna, Austria: R Foundation for Statistical Computing.

\hypertarget{ref-raffa2008}{}
Raffa KF, Aukema BH, and Bentz BJ \emph{et al.} 2008. Cross-scale
Drivers of Natural Disturbances Prone to Anthropogenic Amplification:
The Dynamics of Bark Beetle Eruptions. \emph{BioScience} \textbf{58}:
501--17.

\hypertarget{ref-rouse1973}{}
Rouse W, Haas RH, Deering W, and Schell JA. 1973. MONITORING THE VERNAL
ADVANCEMENT AND RETROGRADATION (GREEN WAVE EFFECT) OF NATURAL
VEGETATION. Greenbelt, MD, USA: Goddard Space Flight Center.

\hypertarget{ref-roussel2019a}{}
Roussel J-R. 2019. lidRplugins: Extra functions and algorithms for lidR
package.

\hypertarget{ref-roussel2019}{}
Roussel J-R, documentation) DA( the, and improved catalog features) FDB(
bugs, and lassnags) ASM(. 2019. lidR: Airborne LiDAR Data Manipulation
and Visualization for Forestry Applications.

\hypertarget{ref-shin2018}{}
Shin P, Sankey T, Moore M, and Thode A. 2018. Evaluating Unmanned Aerial
Vehicle Images for Estimating Forest Canopy Fuels in a Ponderosa Pine
Stand. \emph{Remote Sensing} \textbf{10}: 1266.

\hypertarget{ref-stephenson1998}{}
Stephenson N. 1998. Actual evapotranspiration and deficit: Biologically
meaningful correlates of vegetation distribution across spatial scales.
\emph{Journal of Biogeography} \textbf{25}: 855--70.

\hypertarget{ref-usdafs2019}{}
USDAFS. 2019. Press Release: Survey finds 18 million trees died in
California in
2018\url{https://www.fs.usda.gov/Internet/FSE_DOCUMENTS/FSEPRD609321.pdf}.
Viewed 22 Feb 2019.

\hypertarget{ref-vega2014}{}
Vega C, Hamrouni A, and El Mokhtari S \emph{et al.} 2014. PTrees: A
point-based approach to forest tree extraction from lidar data.
\emph{International Journal of Applied Earth Observation and
Geoinformation} \textbf{33}: 98--108.

\hypertarget{ref-wood2016}{}
Wood SN, Pya N, and Säfken B. 2016. Smoothing Parameter and Model
Selection for General Smooth Models. \emph{Journal of the American
Statistical Association} \textbf{111}: 1548--63.

\hypertarget{ref-young2017}{}
Young DJN, Stevens JT, and Earles JM \emph{et al.} 2017. Long-term
climate and competition explain forest mortality patterns under extreme
drought. \emph{Ecology Letters} \textbf{20}: 78--86.

\hypertarget{ref-zhang2016}{}
Zhang W, Qi J, and Wan P \emph{et al.} 2016. An Easy-to-Use Airborne
LiDAR Data Filtering Method Based on Cloth Simulation. \emph{Remote
Sensing} \textbf{8}: 501.


\end{document}
